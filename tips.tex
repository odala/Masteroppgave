
\chapter{Good to Remember}
% Remark: If you want a shorter chapter or section title to appear in the Table of Contents and in the headings of the chapter, you just include the short title in square brackets before the title of the chapter/section.

\section{Example of Setup}
\begin{itemize}
\item Introduction
    \begin{enumerate}
    \item{Research / Problem}
    \item{Importance of Resarch}
    \item{Significant Prior Research}
    \item{Research Approach or Methodology}
    \item{Limitations and Key Assumptions}
    \item{Potential Outcomes of Research and Importance of Each}
    \end{enumerate}
\item{Theory}
\item{Analytic Solution}
\item{Numerical Solution}
\item{Summary and Outlook}
\end{itemize}

\section{Simple Equations}

\index{angels}The number of angels\nomenclature{$N$}{number of angels} are given by
\begin{equation}
N = \sigma * A
\end{equation}
where $A$ is the total area and $\sigma$ is the total mass of angels per unit area.\nomenclature{$A$}{total area}\nomenclature{$\sigma$}{The total mass of angels per unit area} It is useful to note that \ac{BMI} is important. And that \ac{BMI} is important.

\section{Figures and Tables}
% Remember that all figures and tables shall be referred to and explained/discussed in the text. If a figure/table is not referred to in the text, it shall be deleted from the report.

%%===================================
% Figure.
\begin{figure}[htbp]
    \centering
    \includegraphics[width=1.0\linewidth]{unknown.png}
    \caption[\todo{Caption.}]{\todo{Caption.}}
    \label{fig:add_label}
\end{figure}
Fig.~\ref{fig:add_label}

%%===================================
% Figure with subfigures.
\begin{figure}[htbp]
    \centering
    \begin{subfigure}[0.5\textwidth]
        \centering
        \includegraphics[width=\linewidth]{unknown.png}
        \caption{\todo{Caption.}}
    \end{subfigure}
    \hfill
    \begin{subfigure}[0.5\textwidth]
        \centering
        \includegraphics[width=\linewidth]{unknown.png}
        \caption{\todo{Caption.}}
    \end{subfigure}
    \caption[\todo{Caption.}]{\todo{Caption.}}
    \label{fig:add_label}
\end{figure}
Fig.~\ref{fig:add_label}

%%===================================
% Figure that spans multiple pages.
\begin{figure}[htbp]
    \centering
    \subfigure[\todo{Subcaption.}]{\includegraphics[width=0.4\linewidth]{unknown.png}}
    \hfill
    \subfigure[\todo{Subcaption.}]{\includegraphics[width=0.4\linewidth]{unknown.png}}
    \caption[\todo{Caption}]{\todo{Caption}}\label{fig:add_label}
\end{figure}
\begin{figure}[htbp]
\ContinuedFloat
    \centering
    \subfigure[\todo{Subcaption.}]{\includegraphics[width=0.4\linewidth]{unknown.png}}
    \hfill
    \subfigure[\todo{Subcaption.}]{\includegraphics[width=0.4\linewidth]{unknown.png}}
    \captionsetup{list=no}
    \caption{\emph{(continued)}}
\end{figure}
Fig.~\ref{fig:add_label}
%%===================================
\begin{figure}
    \centering
    \begin{subfigure}[t]{\textwidth}
          \begin{minipage}[t]{0.49\linewidth}
            \centering
            \includegraphics[width=\linewidth]{unknown.png}
          \end{minipage}
          \hfill
          \begin{minipage}[t]{0.49\linewidth}
            \centering
            \includegraphics[width=\linewidth]{unknown.png}
          \end{minipage}
        \caption{\todo{Add caption}}\label{fig:add_label}
    \end{subfigure}
    \par\bigskip
    \begin{subfigure}[t]{\textwidth}
          \begin{minipage}[t]{0.49\linewidth}
            \centering
            \includegraphics[width=\linewidth]{unknown.png}
          \end{minipage}
          \hfill
          \begin{minipage}[t]{0.49\linewidth}
            \centering
            \includegraphics[width=\linewidth]{unknown.png}
          \end{minipage}
        \caption{\todo{Add caption}}\label{fig:add_label}
    \end{subfigure}
    \caption[\todo{Add caption.}]{\todo{Add cation.}}\label{fig:add_label}
\end{figure}

\begin{figure}[htbp]
\ContinuedFloat
    \centering
    \begin{subfigure}[t]{\textwidth}
          \begin{minipage}[t]{0.49\linewidth}
            \centering
            \includegraphics[width=\linewidth]{unknown.png}
          \end{minipage}
          \hfill
          \begin{minipage}[t]{0.49\linewidth}
            \centering
            \includegraphics[width=\linewidth]{unknown.png}
          \end{minipage}
        \caption{\todo{Add caption}}\label{fig:add_label}
    \end{subfigure}
    \captionsetup{list=no}
    \caption{\emph{(continued)}}
\end{figure}
%%===================================

\section{Copying Figures and Tables}
% In some cases, it may be relevant to include figures and tables from from other publications in your report. 
% This can be a direct copy or that you retype the table or redraw the figure. In both cases, you should include a reference to the source in the figure or table caption. The caption might then be written as: Figure/Table xx: The caption text is coming here (Rausand and Høyland, 2004).

% In other cases, you get the idea from a figure or table in a publication, but modify the figure/table to fit your purpose. If the change is significant, your caption should have the following format: Figure/Table xx: The caption text is coming here (adapted from Rausand and Høyland, 2004

\section{Plagiarism}
% Plagiarism is defined as “use, without giving reasonable and appropriate credit to or acknowledging the author or source, of another person’s original work, whether such work is made up of code, formulas, ideas, language, research, strategies, writing or other form”, and is a very serious issue in all academic work. You should adhere to the following rules:
% • Give proper references to all the sources you are using as a basis for your work. The references should be give to the original work and not to newer sources that mention the original sources.
% • You may copy paragraphs up to 50 words when you include a proper reference. In doing so, you should place the copied text in inverted commas (i.e., “Copied text follows . . . ”). Another option is to write the copied text as a (idented) quotation.

\section{Sentence connectors (Bindeord)}

Sentence connectors are used to link ideas from one sentence to the next and to give paragraphs coherence. Sentence connectors perform different functions and are placed at the beginning of a sentence. They are used to introduce, order, contrast, sequence ideas, theory, data etc. The following table lists useful connectors.

\subsection{Logical / sequential order}
\begin{itemize}
    \item Firstly, secondly, thirdly etc
    \item Next, last, finally
    \item In addition
    \item Furthermore
    \item Also
    \item At present / presently
\end{itemize}

\subsection{Time}
\begin{itemize}
    \item Meanwhile
    \item Presently
    \item At last
    \item Finally
    \item Immediately
    \item Thereafter
    \item At that time
    \item Subsequently
    \item Eventually
    \item Currently
    \item In the meantime
    \item In the past
    \item In recent years
\end{itemize}

\subsection{Place}
\begin{itemize}
    \item There
    \item Here
    \item Beyond
    \item Nearby
    \item Next to
    \item At that point
    \item Opposite to
    \item Adjacent to
    \item On the other side
    \item In the front
    \item In the back
\end{itemize}

\subsection{Order of importance}
\begin{itemize}
    \item Most / more importantly
    \item Most significantly
    \item Above all
    \item Primarily
    \item It is essential / essentially
\end{itemize}


\subsection{Contrast}
\begin{itemize}
    \item However
    \item Nevertheless
    \item On the other hand
    \item On the contrary
    \item Even so
    \item Notwithstanding
    \item Alternatively
    \item At the same time
    \item Though
    \item Otherwise
    \item Instead
    \item Nonetheless
    \item Conversely
    \item By (in) comparison
    \item In contrast
\end{itemize}


\subsection{Result}
\begin{enumerate}
    \item As a result
    \item As a consequence / In consequence
    \item Therefore
    \item Thus
    \item Consequently
    \item Hence
    \item Accordingly
    \item Thereupon
    \item So
    \item Then
\end{enumerate}


\subsection{Comparison}
\begin{itemize}
    \item Similarily
    \item Comparable
    \item In the same way
    \item Likewise
    \item As with
    \item Equally
    \item Just as ... so too ...
    \item A similar $x$
    \item Another $x$ like
\end{itemize}


\subsection{Reason}
\begin{itemize}
    \item The cause of
    \item The reason for
\end{itemize}

\subsection{Example}
\begin{itemize}
    \item For example
    \item For instance
    \item That is
    \item Such as
    \item As revealed by
    \item Illustrated by
    \item Specifically
    \item In particular
    \item For one thing
    \item This can be seen in
    \item An instance of this
\end{itemize}

\subsection{Addition}
\begin{itemize}
    \item further
    \item furthermore
    \item moreover
    \item in addition
    \item additionally
    \item then
    \item also
    \item too
    \item besides
    \item again
    \item equally important
    \item first, second, ...
    \item finally, last
\end{itemize}

\subsection{Summary}
\begin{itemize}
    \item In short
    \item On the whole
    \item In other words
    \item To be sure
    \item Clearly
    \item Anyway
    \item In sum
    \item After all
    \item In general
    \item It seems
    \item In brief
\end{itemize}

% --- Eksempler på gode bindeord:
%     Årsaksforhold: fordi, derfor, dersom, som følge av, følgelig, på grunn av, ettersom, siden, hvis, altså.
%     Tidsforhold: da, så, slik, således, deretter, etterpå, mens, først, samtidig, tidligere.
%     Motsetninger: men, derimot, imidlertid, på tross av, selv om, ikke desto mindre, likevel, tvert imot.
%     Likhet/symmetri: og, dessuten, videre, blant annet, til og med, også, på samme måte, eller.



\begin{figure}
    \centering
    \begin{subfigure}[t]{0.4\linewidth}
        \centering
        \begin{tikzpicture}[remember picture]
          \node[anchor=south west,inner sep=0] (imageA) {\includegraphics[width=\linewidth]{subBb_twins.png}};
          \begin{scope}[x={(imageA.south west)},y={(imageA.north east)}]
            \node[coordinate] (A) at (0.1\linewidth, 0.9\linewidth) {};
          \end{scope}
        \end{tikzpicture}
    \caption{}\label{fig:subBb_twins_illustration}
    \end{subfigure}
    \hfill
    \begin{subfigure}[t]{0.5333\linewidth}
    \centering
        \begin{tikzpicture}[remember picture]
          \node[anchor=south west,inner sep=0] (imageB) {\includegraphics[width=\linewidth]{subBb_sem_01_m005.jpg}};
          \begin{scope}[x={(imageB.south west)},y={(imageB.north east)}]
            \node[coordinate] (B) at (0.5,0.5) {};
          \end{scope}
        \end{tikzpicture}
    \caption{}\label{fig:subBb_sem_twins}
    \end{subfigure}
    \caption[]{}\label{}
\end{figure}
% Draw arrow.
\begin{tikzpicture}[remember picture,overlay]
  \draw[->, red, line width=0.5mm ] (B) -- (A);
\end{tikzpicture}
% [ draw=black,solid,line width=2mm,fill=black,  preaction={-triangle 90,thin,draw,shorten >=-1mm}]

