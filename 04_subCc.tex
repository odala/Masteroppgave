\clearpage
%%=========================================
\section{MCT Film Grown by MBE on  Substrate C}\label{sec:subCc}

A film of \acl{mct} was grown on substrate C by \ac{mbe}. 

\begin{figure}[htbp]
    \centering
    \mySubfigure{0.49\textwidth}{unknown.png}[fig:subCc_om_centre][angle=180]
    \hfill
    \mySubfigure{0.49\textwidth}{unknown.png}[fig:subCc_om_edge][angle=180]
    \caption[Bright field microscopy images of \ac{mct} film grown by \ac{mbe} on substrate AC]{Bright field microscopy images of \ac{mct} film grown by \ac{mbe} on (211)B-oriented substrate C: \subref{fig:subCc_om_centre} Near the centre; and \subref{fig:subCc_om_edge} near the right edge.}
    \label{fig:subCc_om}
\end{figure}

%%=========================================
\subsection{Particles}

\citet{selvig2007defects} describe the irregularly shaped defects with sizes extending from \SI{10}{\micro\metre} to a few hundred microns as high temperature voids, see Fig.~\ref{}, and the ones with size less than \SI{10}{\micro\metre} as microvoids, see Fig.~\ref{}. These types of defects are typically formed during the growth of the material and the density is dependent on the deviation from ideal growth conditions. High temperature voids are formed at substrate temperatures higher than the \ce{Te}-phase limit, while microvoids are formed at low substrate temperature.

\todo{Sammenlign med telling av tellurium precipitates.}

%%=========================================
\subsection{Composition and Thickness}

\Iac{ftir} transmission spectrum was recorded from one spot in the centre of the \ac{mct} film grown by \ac{mbe} on substrate C. The \ac{ir} transmittance was between \SI{50}{\percent} and \SI{63}{\percent} in the wavenumber range between \SIrange{400}{1400}{\centi\metre^{-1}}, see Fig.~\ref{fig:subCc_ftir}. The cut-on wavenumber was at \SI{1564.7}{\centi\metre^{-1}}, which corresponds to a wavelength of \SI{6.39}{\micro\metre}. Consequently, the $x$-value was \SI{0.224}{} according to the formula by \citet{bricexxxxtttt}. The thickness of the \ac{mct} film was calculated to be \SIrange{11.0}{11.6}{\micro\metre} by using Eq.~\ref{eq:ftir_thickness}.


\todo{FTIR: correlate with polishing grit or donut density.}

\begin{figure}[htbp]
    \centering
    \mySubfigure{0.60175438596\linewidth}{CMT801_ftir_spectra.png}[fig:subCc_ftir_spectra]
    \hfill
    \mySubfigure{0.37824561403\linewidth}{CMT801_ftir_transmission_at_k500cm-1.png}[fig:subCc_ftir_map_500cm-1]
    \caption[\Ac{ftir} measurement from one spot on the \ac{mct} film grown on substrate C.]{\Ac{ftir} measurements recorded from one spot on the $\SI{15}{\milli\metre}\times\SI{15}{\milli\metre}$ \ac{mct} film grown on substrate C: \subref{fig:subCc_ftir_spectra} Transmission spectrum; \subref{fig:subCc_ftir_map_500cm-1} transmission map at wavenumber $k=\SI{500}{\centi\metre^{-1}}$ showing the transmittance $T$ in percentage of incoming light that is transmitted through at the point.}\label{fig:subCc_ftir}
\end{figure}

%%=========================================
\subsection{Impurity Analysis}

\begin{comment}
\begin{table}[htbp]
    \centering
    \caption[\Ac{eds} impurity analysis of \ac{mct} film grown by \ac{mbe} on substrate C.]{Results of the \ac{eds} impurity analysis at three different locations on the $\SI{15}{\milli\metre}\times\SI{15}{\milli\metre}$  \ac{mct} film grown by \ac{mbe} on (211)B-oriented substrate C (atomic concentration \%). The X-ray signal is acquired from $\SI{1270}{\micro\metre}\times\SI{890}{\micro\metre}$ areas near the centre, upper edge, and upper left corner.}\label{tab:subBc_eds_analysis}
   \begin{tabu} to 1.0\textwidth { X[1.85, r] X[1.125,c] X[1.125,c] X[1.125,c] X[1.125,c] X[1.125,c] X[1.125,c] }
        \hline
            & \textbf{\ce{Te}} (at.\%) & \textbf{\ce{Hg}} (at.\%) & \textbf{\ce{Cd}} (at.\%) & \textbf{\ce{C} } (at.\%) & \textbf{\,\ce{O}\,} (at.\%) & \textbf{\ce{Al}} (at.\%) \\
        \hline
        Near centre & \SI{44,36}{} & \SI{31,14}{} & \SI{12,19}{} & \SI{11,30}{} & \SI{0,86}{} & \SI{0,16}{} \\
        Near edge & \SI{44,12}{} & \SI{32,26}{} & \SI{10,81}{} & \SI{11,49}{} & \SI{1,14}{} & \SI{0,18}{} \\
        Near corner & \SI{44,26}{} & \SI{32,58}{} & \SI{10,71}{} & \SI{11,23}{} & \SI{1,03}{} & \SI{0,19}{}  \\
        \hline
    \end{tabu}
\end{table}
\end{comment}

%%=========================================