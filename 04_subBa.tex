\clearpage
%%=========================================
\section{Surface Analysis of As-Received Substrate B}\label{sec:subBa}
% Substrate B
Substrate B is from an alternative source (vendor B) and had only been roughly polished after being sliced from a bulk crystal. In previous work \citep{lauten2017characterisation}, the as-received (111)B-oriented substrate B was characterised for polishing damage, defects, and residual particles using optical microscopy and \ac{sem} with \ac{eds}. The results are reiterated in this section to better present the full scope of the study. In addition to the previously used methods, \ac{afm}, near-\ac{ir} transmission microscopy, and \ac{ftir} were used to study the as-received substrate.

Fig.~\ref{fig:subBa_om_df} shows typical dark field images from the surface of substrate B at the corner, edge, and centre of the substrate. Polishing scratches in all directions and with varying width can be observed, both at the centre and towards the edges, on the surface of substrate B. Big particles and morphological defects with a diameter between \SI{10}{\micro\metre} and \SI{50}{\micro\metre}, as well as smaller particles and morphological defects with diameter \SI{<10}{\micro\metre}, are present on the surface. By counting the number of spots in the dark field image, the particle and morphological defect density is estimated to be \SI{1e5}{\centi\metre^{-2}}, both at the centre and at the edges of the surface of substrate B. Also, here the particle and morphological defect density refers to the sum of all light scattering objects \SI{>0.5}{\micro\metre} in size.
% --- Measured with tolerance of 10 using ImageJ. One pixel are 2250/2048 um.
%       Centre: 5739 partikler på 2048x1536 --> 151151 particles per cm^2
%       Edges: 3927 partikler på 1877x1522 --> 113888 particles per cm^2
%       Corner: 5327 partikler på 1848x1486 --> 158000 particles per cm^2

\begin{figure}[htbp]
    \centering
    \mySubfigure{\linewidth}{LM_DF_C389523A_M005_centre.jpg}[fig:subBa_om_df_centre]
    \par\bigskip
    \mySubfigure{0.49\linewidth}{LM_DF_C389523A_M005_centreEdge.jpg}[fig:subBa_om_df_edge]
    \hfill
    \mySubfigure{0.49\linewidth}{LM_DF_C389523A_M005_corner.jpg}[fig:subBa_om_df_corner]
    \caption[Dark field images of substrate B.]{Dark field images of substrate B captured through the optical microscope Leica DM RXA2 at the \subref{fig:subBa_om_df_centre} centre; \subref{fig:subBa_om_df_edge} edge; and \subref{fig:subBa_om_df_corner} corner of the substrate.}
    \label{fig:subBa_om_df}
\end{figure}

The difference between substrate A and substrate B is significant. Substrate B has a density of particles and morphological defects that is 100-1000 times larger than on substrate A. Some of this can be explained by the more thorough preparation that substrate A has been subjected to. Substrate A has had a final polishing and an etch before being delivered, while substrate B is roughly polished and particles on the surface have not been removed.

A comparison between the dark field image and a \ac{sem} image of the same area of substrate B, as seen in Fig.~\ref{fig:LM_SEM_C3895}, reveals that the brightest spots in the dark field image are from cavities in the substrate surface and that particles as small as \SI{0.5}{\micro\metre} can be seen in the dark field image. The dark stains, on the other hand, are not visible in the dark field image. This indicates that the dark stains do not have sharp edges or other pointy features.

\begin{figure}[htbp]
    \centering
    %\mySubfigure[Dark field optical microscope image.]{\includegraphics[width=1\linewidth]{C-3895-23A_centre_LM.jpg}\label{fig:C-3895-23A_centre_LM}}
    %\mySubfigure[SEM image at a magnification of 60$\times$.]{\includegraphics[width=1\linewidth]{C-3895-23A_centre_SEM.jpg}\label{fig:C-3895-23A_centre_SEM}}
    \mySubfigure{0.36\linewidth}{C-3895-23A_edge_LM_500_overview.jpg}[fig:C-3895-23A_edge_LM_500_overview]
    \hfill
    \mySubfigure{0.60\linewidth}{C-3895-23A_edge_SEM_500_overview.png}[fig:C-3895-23A_edge_SEM_500_overview]
    \caption[Comparison of dark field microscopy and \ac{sem} images.]{Comparison of \subref{fig:C-3895-23A_edge_LM_500_overview} a dark field microscopy image captured through the optical microscope Leica DM RXA2 and \subref{fig:C-3895-23A_edge_SEM_500_overview} \iacf{sem} image taken at the same location on the as-received substrate B as the dark field image.}
    \label{fig:LM_SEM_C3895}
\end{figure}

A \ac{sem} image mapping of substrate B was done at points in a grid of $11\times11$ grid points. The distance between neighbouring points was \SI{2.60}{\milli\metre} and the distance from the edge to the outer points was \SI{2.00}{\milli\metre}. A low magnification \ac{sem} image (60$\times$) and a higher magnification image (500$\times$) were acquired at every grid point. %This mapping will be used in future studies of substrate B when correlation between defect density and preparation methods is going to be measured.

%A particle and morphological defect density of features \SI{>0.5}{\micro\metre} of \SI{1e5}{\centi\metre^{-2}} at both the centre and at the edges of the substrate was observed. Substrate B had polishing scratches that were between 10 and \SI{100}{\nano\metre} wide. A large amount of voids with sizes ranging from \SI{5}{} to \SI{100}{\micro\metre} were observed on the surface of substrate B. 

%Most of the particles \SI{>1}{\micro\metre} on the surface of substrate B were \ce{CdZnTe} particles with lengths of between \SI{50}{} and \SI{100}{\micro\metre} that could be debris from the cutting of the substrate, but some carbon based particles was also observed with lengths of \SI{25}{\micro\metre}. Most of the particles \SI{<1}{\micro\metre} on the surface of substrate B were residual \ce{Al2O3} and \ce{SiO2} polishing grit with diameter of between \SI{50}{} and \SI{100}{\nano\metre}.  In addition, iron particles, which could be possible remainders of polishing slurry, and particles containing \ce{Na} and \ce{Cl}, which could be possible remainders of polishing slurry cleaner, were observed on the surface of substrate B.

%%=========================================
%\section{Particles and Defects on Substrate B}

One of the \ac{sem} images from the grid at the middle of substrate B taken at low magnification (60$\times$), see Fig.~\ref{fig:C-3895-23A_F08_x060}, shows some bright stains at the top of the image, a dark spot down in the right corner, and a lot of bright spots all over. The \ac{sem} image of the same area at higher magnification (500$\times$), see Fig.~\ref{fig:C-3895-23A_F08_x500}, reveals that there are surface scratches, dark stains of different sizes and some even smaller bright spots on the surface. These features will be described in the following by, among other methods, \ac{sem} images at even higher magnifications.

\begin{figure}[htbp]
    \centering
    \mySubfigure{0.49\linewidth}{C-3895-23A_F08_x060.png}[fig:C-3895-23A_F08_x060]
    \hfill
    \mySubfigure{0.49\linewidth}{C-3895-23A_F08_x500.png}[fig:C-3895-23A_F08_x500]
    \caption[\Ac{sem} images of a typical area in the middle of substrate B.]{\Ac{sem} images of a typical area in the middle of substrate B at \subref{fig:C-3895-23A_F08_x060} $60\times$ magnification; and \subref{fig:C-3895-23A_F08_x500} $500\times$ magnification.}
    \label{fig:SEM_C389523_overview}
\end{figure}

\subsection{Particles and Surface Features}

Seven different types of particles and surface features were observed on the surface of substrate B. These are shown in Fig.~\ref{fig:subBa_sem_w_eds}.
%\mySubfigure[SEM.]{0.44\linewidth}{C-3895-23_09_m001.jpg}
    %\mySubfigure[SEM.]{0.44\linewidth}{C-3895-23A_edx7_m002.jpg}
    %\mySubfigure[SEM.]{0.44\linewidth}{C-3895-23A_edx7_m003.jpg}
\begin{figure}[htbp]
    \centering
    \begin{subfigure}[t]{\textwidth}
        \caption{}\label{fig:subBa_polishing-grit_alumina}
          \begin{minipage}[t]{0.43\linewidth}
            \centering
            \includegraphics[width=\linewidth]{alumina02_sem.png}
          \end{minipage}
          \hfill
          \begin{minipage}[t]{0.43\linewidth}
            \centering
            \includegraphics[width=\linewidth]{alumina02_eds.jpg}
          \end{minipage}
          \begin{minipage}[t]{0.11\linewidth}
            \centering
            \atomicTable[&][&][&]
          \end{minipage}
    \end{subfigure}%
    \par\bigskip
    \begin{subfigure}[t]{\textwidth}
        \caption{}\label{fig:subBa_polishing-grit_silica}
          \begin{minipage}[t]{0.43\linewidth}
            \centering
            \includegraphics[width=\linewidth]{subB_silica_sem.png}
          \end{minipage}
          \hfill
          \begin{minipage}[t]{0.43\linewidth}
            \centering
            \includegraphics[width=\linewidth]{subB_silica_eds.jpg}
          \end{minipage}
          \begin{minipage}[t]{0.11\linewidth}
            \centering
            \atomicTable[&][&][&]
          \end{minipage}
    \end{subfigure}%
    \par\bigskip
    \begin{subfigure}[t]{\textwidth}
        \caption{}\label{fig:SEM_B_particulates_eds}
          \begin{minipage}[t]{0.43\linewidth}
            \centering
            \includegraphics[width=\linewidth]{C-3895-23Af2_m002.png}
          \end{minipage}
          \hfill
          \begin{minipage}[t]{0.43\linewidth}
            \centering
            \includegraphics[width=\linewidth]{CdZnTe_eds.jpg}
          \end{minipage}
          \begin{minipage}[t]{0.11\linewidth}
            \centering
            \atomicTable[&][&][&]
          \end{minipage}
          %\begin{minipage}[t]{0.49\linewidth}
          %  \centering
          %  \includegraphics[width=\linewidth]{CdZnTe_eds_substrate.jpg}
          %\end{minipage}
    \end{subfigure}%
    \caption[\Ac{sem} images, \ac{eds} spectra, and \ac{eds} atomic compositions of one void and six different types of particles found on as-received substrate B.]{High resolution \ac{sem} images of one void and six different types of particles found on the as-received substrate B and the corresponding \ac{eds} spectra and atomic compositions: \subref{fig:subBa_polishing-grit_alumina} alumina (\ce{Al2O3}) polishing grit; \subref{fig:subBa_polishing-grit_silica} silica (\ce{SiO2}) polishing grit; \subref{fig:SEM_B_particulates_eds} \ac{czt} paticle; \subref{fig:subBa_particle_carbon} carbon-based particle; \subref{fig:EDS_NaClO} \ce{NaClO}; \subref{fig:SEM_C389523_void_eds} void; and \subref{fig:subBa_partice_Fe} iron (\ce{Fe}).}\label{fig:subBa_sem_w_eds}
\end{figure}
%
\begin{figure}[htbp]
\ContinuedFloat
    \centering
    \begin{subfigure}[t]{\textwidth}
        \caption{}\label{fig:subBa_particle_carbon}
          \begin{minipage}[t]{0.43\linewidth}
            \centering
            \includegraphics[width=\linewidth]{C-3895-23A_tuning_05.png}%{carbon_eds_sem.jpg} %{C-3895-23_02_m002.jpg}%{C-3895-23A_tuning_05.jpg}
          \end{minipage}
          \hfill
          \begin{minipage}[t]{0.43\linewidth}
            \centering
            \includegraphics[width=\linewidth]{carbon_eds.jpg}
          \end{minipage}
          \begin{minipage}[t]{0.11\linewidth}
            \centering
            \atomicTable[&][&][&]
          \end{minipage}
    \end{subfigure}%
    \par\bigskip
    \begin{subfigure}[t]{\textwidth}
    \caption{}\label{fig:EDS_NaClO}
          \begin{minipage}[t]{0.43\linewidth}
            \centering
            \includegraphics[width=\linewidth]{eds_NaOCl2_C-3895-23A_edx8_m007.png}
          \end{minipage}
          \hfill
          \begin{minipage}[t]{0.43\linewidth}
            \centering
            \includegraphics[width=\linewidth]{eds_NaClO2.jpg}
          \end{minipage}
          \begin{minipage}[t]{0.11\linewidth}
            \centering
            \atomicTable[&][&][&]
          \end{minipage}
    \end{subfigure}%
    \par\bigskip
    \begin{subfigure}[t]{\textwidth}
        \caption{}\label{fig:SEM_C389523_void_eds}
          \begin{minipage}[t]{0.43\linewidth}

            \centering
            \includegraphics[width=\linewidth]{C-3895-23_09_m001.png}%C-3895-23A_edx3_m001.jpg}
          \end{minipage}
          \hfill
          \begin{minipage}[t]{0.43\linewidth}
            \centering
            \includegraphics[width=\linewidth]{C-3895-23A_edx3_m001_eds.jpg}
          \end{minipage}
          \begin{minipage}[t]{0.11\linewidth}
            \centering
            \atomicTable[&][&][&]
          \end{minipage}
    \end{subfigure}%
    \captionsetup{list=no}
    \caption{\emph{(continued)}}
\end{figure}
%
\begin{figure}[htbp]
\ContinuedFloat
    \centering
    \begin{subfigure}[t]{\textwidth}
    \caption{}\label{fig:subBa_partice_Fe}
          \begin{minipage}[t]{0.43\linewidth}
            \centering
            \includegraphics[width=\linewidth]{eds_Fe_C-3895-23A_edx5_m004.png}
          \end{minipage}
          \hfill
          \begin{minipage}[t]{0.43\linewidth}
            \centering
            \includegraphics[width=\linewidth]{eds_Fe.jpg}
          \end{minipage}
          \begin{minipage}[t]{0.11\linewidth}
            \centering
            \atomicTable[&][&][&]
          \end{minipage}
    \end{subfigure}%
    \captionsetup{list=no}
    \caption{\emph{(continued)}}
\end{figure}

%%=====
\subsubsection{Residual Polishing Grit (alumina and silica)}
Fig.~\ref{fig:subBa_polishing-grit_area1} displays an area on the surface where there are accumulations of small particles. Fig.~\ref{fig:subBa_polishing-grit_area2} display one of these accumulations at greater magnification. The large particles appear to be agglomerations of smaller particles that are between \SI{50}{\nano\metre} and \SI{100}{\nano\metre} in diameter. The typical width of the particle agglomerations is \SI{0.5}{}-\SI{3}{\micro\metre}. The attraction between the small particles can be explained by electrostatic attractive forces between the particles due to different surface charge \citep{allen2001review}.

\begin{figure}[htbp]
    \centering
        \mySubfigure{0.49\linewidth}{SEM_C-3895-23A_a_m001.png}[fig:subBa_polishing-grit_area1]
        \hfill
        \mySubfigure{0.49\linewidth}{C-3895-23Ab_m003.png}[fig:subBa_polishing-grit_area2]
    \caption[\Ac{sem} images of an accumulaton of polishing grit on substrate B.]{\Ac{sem} images of an accumulation of polishing grit on the as-received substrate B at a magnification of \subref{fig:subBa_polishing-grit_area1} $200\times$ and \subref{fig:subBa_polishing-grit_area2} $15000\times$.}\label{fig:subBa_polishing-grit_area}
\end{figure}

An \ac{eds} spectrum of the particle reveals that the piece is composed of alumina oxide, \ce{Al2O3}, also known as alumina, see Fig.~\ref{fig:subBa_polishing-grit_alumina}. The corresponding \ac{sem} image of residual polishing grit is shown next to the spectrum. The presence of alumina can be explained by the frequent use of alumina as an abrasive in polishing slurries for semiconducting material.

The slurry polishes and flattens the substrate through mechanical action. The alumina particles in the slurry are suspended throughout the bulk of the medium because of the negative charges that are incorporated on the alumina particle surface and make the particles repel each other. \todo{...} can explain the observations of agglomerations of alumina.

Fig.~\ref{fig:subBa_polishing-grit_silica} displays an area on the surface where there are an accumulation particles that look similar to the alumina agglomerations. The largest particle in the image has a diameter of \SI{600}{\nano\metre}. An \ac{eds} spectrum of the particle reveals that the piece is composed of silicon oxide, \ce{SiO2}, also known as silica. As mentioned, silica is a frequently used abrasive in polishing slurry.

%%=====
\subsubsection{\Ac{czt}}
%Page 5-6 in eds report 1-10
Fig.~\ref{fig:SEM_B_particulates} shows \ac{sem} images of large pieces of material on the substrate surface. The size of the pieces is typically between \SI{50}{\micro\metre} and \SI{100}{\micro\metre}. By comparing the \ac{eds} spectrum of the particle with the spectrum of the substrate surface, see Fig.~\ref{fig:SEM_B_particulates_eds}, it is seen that they are essentially the same. This indicates that the pieces consist of the same material as the substrate and that they could be debris from the polishing or cutting of the substrate.

\begin{figure}[htbp]
    \centering
          \begin{minipage}[t]{0.49\linewidth}
            \centering
            \includegraphics[width=\linewidth]{C-3895-23_03.png}
          \end{minipage}
          \hfill
          \begin{minipage}[t]{0.49\linewidth}
            \centering
            \includegraphics[width=\linewidth]{C-3895-23A_edx1_m006.png}
          \end{minipage}
        \caption[\Ac{sem} images of \ac{czt} particles on as-revceived substrate B.]{\Ac{sem} images of \ac{czt} particles on the as-revceived substrate B.}\label{fig:SEM_B_particulates}  
\end{figure}


%%=====
\subsubsection{Carbon-based particle}
Dark particles with typical size between \SI{20}{} and \SI{30}{\micro\metre} were observed on the substrate surface. The \ac{eds} spectrum of this particle shows a high intensity from the carbon signal, see Fig.~\ref{fig:subBa_particle_carbon}. The particles could be residue from mounting wax, which is used to hold the substrate while it is being cut and polished. There are some small peaks from silicon and aluminium as well, but these are probably from the residual polishing grit that can be seen in on the surface of the carbon-based particle.
%Page 8+12 in eds report 1-10

%%=====
\subsubsection{\ce{NaClO}}
An area with lots of circular particles with diameter between \SI{100}{\nano\metre} and \SI{1}{\micro\metre} is observed near one of the edges of substrate B, see Fig.~\ref{fig:eds_NaOCl_overview}. The area is separated from the rest of the substrate by a dark borderline. An \ac{eds} spectrum of one particle with diameter of \SI{1}{\micro\metre}, see Fig.~\ref{fig:EDS_NaClO}, reveals that the particle consists of \ce{Na} and \ce{Cl}. The particles could be \ce{NaClO} which is used after polishing as a standard cleaner to remove polishing slurry particles \citep{benson2015as-received}. The dark borderline was not possible to get quantified with \ac{eds}, but it could be a residue of the cleaning solution.

\begin{figure}
    \centering
    \includegraphics[width=0.48\linewidth]{eds_NaOCl_overview_C-3895-23A_edx8_m008.png}
    \caption[\Ac{sem} image of \ce{NaClO} particles on substrate B.]{\Ac{sem} image of \ce{NaClO} particles on substrate B.}
    \label{fig:eds_NaOCl_overview}
\end{figure}


%%=====
\subsubsection{Voids}
%Page b1 in eds report 1-10
Irregular shaped voids were observed all over the surface of substrate B, see Fig.~\ref{fig:SEM_C389523_voids}. The size of the voids tends to be between \SI{5}{} and \SI{100}{\micro\metre} and \ac{afm} measurements gives that the voids are between \todo{\SI{1.3}{\micro\metre} and \SI{2.5}{\micro\metre}} deep, see Fig.~\ref{subBa_afm_voids}. \Ac{eds} detected \ce{Cd_{0.96}Zn_{0.04}Te} both on the inside edges of the voids and around the voids, see Fig.~\ref{fig:SEM_C389523_void_eds}. This reveals that the voids have the same composition as the substrate surface.

\begin{figure}[htbp]
    \centering
    \begin{subfigure}[t]{\textwidth}
    \caption{}\label{fig:subBa_voids}
          \begin{minipage}[t]{0.49\linewidth}
            \centering
            \includegraphics[width=\linewidth]{C-3895-23A_tuning_04.png}
          \end{minipage}
          \hfill
          \begin{minipage}[t]{0.49\linewidth}
            \centering
            \includegraphics[width=\linewidth]{C-3895-23A_edx3_m002.png}
          \end{minipage}
    \end{subfigure}%
    \par\bigskip
    %\mySubfigure[SEM.]{0.44\linewidth}{C-3895-23A_edx7_m002.jpg}
    %\mySubfigure[SEM.]{0.44\linewidth}{C-3895-23A_edx7_m003.jpg}
    \begin{subfigure}[t]{\textwidth}
    \caption{}\label{fig:subBa_microvoids}
          \begin{minipage}[t]{0.49\linewidth}
            \centering
            \includegraphics[width=\linewidth]{C-3895-23A_edx7_m004.png}
          \end{minipage}
          \hfill
          \begin{minipage}[t]{0.49\linewidth}
            \centering
            \includegraphics[width=\linewidth]{C-3895-23A_K01_detail.png}
          \end{minipage}
    \end{subfigure}%
    \caption[\Ac{sem} images of voids on substrate B.]{\Ac{sem} images of \subref{fig:subBa_voids} high temperature voids and \subref{fig:subBa_microvoids} microvoids on the as-received substrate B.}
    \label{fig:SEM_C389523_voids}
\end{figure}

\begin{figure}
    \centering
    \begin{subfigure}[c]{0.032\linewidth}
        \label{fig:subBa_afm_voids_scale}\captionsetup{list=no}
        \includegraphics[width=\linewidth]{subBa_afm_voids_scale.png}
    \end{subfigure}
    \hfill
    \mySubfigure{0.46\linewidth}{subBa_afm_void_170201Topography013.png}[fig:subBa_afm_void]
    \hfill
    \mySubfigure{0.46\linewidth}{subBa_afm_microvoid_170201Topography005.png}[fig:subBa_afm_microvoid]
    \caption[\Ac{afm} measurements of void and microvoid on as-received substrate B.]{\Ac{afm} measurements of \subref{fig:subBa_afm_void} a high temperature void and \subref{fig:subBa_afm_microvoid} a microvoid on the as-received substrate B displayed as images of $\SI{50}{\micro\metre}\times\SI{50}{\micro\metre}$ and $\SI{10}{\micro\metre}\times\SI{10}{\micro\metre}$ areas respectively.}
    \label{fig:subBa_afm_voids}
\end{figure}

Some of the smaller voids have a threefold symmetry as seen in Fig.~\ref{fig:subBa_microvoids}. The larger ones tend to have multiple angular features, as seen in Fig.~\ref{fig:subBa_voids}. \citet{reddy2013cross} speculate that these voids originally was occupied by a tellurium precipitate that have been knocked loose during surface preparation, i.e. substrate polishing in this case, leaving a void on the surface. This theory is supported by the fact that tellurium precipitates are often crystalline and could be the source to the angular features of the voids \citep{wang2008observation}.

The void density was found to be between \SIrange{2e+03}{2e+4}{\centi\metre^{-2}}. The mean void density was \SI{7e+03}{\centi\metre^{-2}} with a standard deviation of \SI{5e+03}{\centi\metre^{-2}}. A graphical representation of the void density at different locations on substrate B can be seen in Fig.~\ref{fig:subBa_densityData_voids}.

\begin{figure}[htbp]
    \centering
        \mySubfigure{0.403\linewidth}{subBa_densityData_voids.png}[fig:subBa_densityData_void_map]
        \hfill
        \mySubfigure{0.577\linewidth}{C-3895-23A_A01_x060.png}[fig:subBa_densityData_void_sem]
    \caption[Map of the void density on the as-received substrate B.]{\subref{fig:subBa_densityData_void_map} A map of the void density at 36 different locations on the as-received $\SI{30}{\milli\metre}\times\SI{30}{\milli\metre}$ substrate B. The void density was observed to vary between \SIrange{2e+03}{2e+4}{\centi\metre^{-2}}. \subref{fig:subBa_densityData_void_sem} An image from the upper right corner of the grid where the void density is highest.}
    \label{fig:subBa_densityData_voids}
\end{figure}

\begin{comment}
The void density was found to be between \SI{1e+02}{\centi\metre^{-2}} and \SI{2e+3}{\centi\metre^{-2}}. The mean void density was \SI{4e+02}{\centi\metre^{-2}} with a standard deviation of \SI{3e+02}{\centi\metre^{-2}}. A graphical representation of the void density at different locations on substrate B can be seen in Fig.~\ref{fig:subBa_densityData_largevoid}.

\begin{figure}[htbp]
    \centering
        \mySubfigure{0.403\linewidth}{subBa_densityData_largevoids.png}[fig:subBa_densityData_largevoid_map]
        \hfill
        \mySubfigure{0.577\linewidth}{C-3895-23A_A01_x060.png}[fig:subBa_densityData_largevoid_sem]
    \caption[Map of the large void density on the as-received substrate B.]{\subref{fig:subBa_densityData_void_map} A map of the large void density at 36 different locations on the as-received $\SI{30}{\milli\metre}\times\SI{30}{\milli\metre}$ substrate B. The large void density was observed to vary between \SI{1e+02}{\centi\metre^{-2}} and \SI{2e+3}{\centi\metre^{-2}}. \subref{fig:subBa_densityData_void_sem} An image from the upper right corner of the grid where the large void density is highest.}
    \label{fig:subBa_densityData_void}
\end{figure}

The microvoid density was found to be between \SI{2e+03}{\centi\metre^{-2}} and \SI{2e+4}{\centi\metre^{-2}}. The mean microvoid density was \SI{6e+03}{\centi\metre^{-2}} with a standard deviation of \SI{5e+03}{\centi\metre^{-2}}. A graphical representation of the microvoid density at different locations on substrate B can be seen in Fig.~\ref{fig:subBa_densityData_microvoid}.
%Voids: Minimum = 1.28e+02. Maximum = 1.91e+03. Mean = 4.20e+02. Standard deviation = 3.26e+02.
%Microvoids: Minimum = 2.21e+03. Maximum = 2.21e+04. Mean = 6.09e+03. Standard deviation = 5.06e+03

\begin{figure}[htbp]
    \centering
        \mySubfigure{0.403\linewidth}{subBa_densityData_microvoids.png}[fig:subBa_densityData_microvoid_map]
        \hfill
        \mySubfigure{0.577\linewidth}{C-3895-23A_A11_x500.png}[fig:subBa_densityData_microvoid_sem]
    \caption[Map of the microvoid density on the as-received substrate B.]{\subref{fig:subBa_densityData_microvoid_map} A map of the microvoid density at 36 different locations on the as-received $\SI{30}{\milli\metre}\times\SI{30}{\milli\metre}$ substrate B. The microvoid density was observed to vary between \SI{2e+03}{\centi\metre^{-2}} and \SI{2e+4}{\centi\metre^{-2}}. \subref{fig:subBa_densityData_microvoid_sem} An image from the upper right corner of the grid where the microvoid density is highest.}
    \label{fig:subBa_densityData_microvoid}
\end{figure}
\end{comment}
%%=====
\subsection{Iron Particle}
A small particle that is \SI{1.5}{\micro\metre} long and \SI{0.6}{\micro\metre} wide can be seen in Fig.~\ref{fig:subBa_partice_Fe} with its corresponding \ac{eds} spectrum. The \ac{eds} spectrum of the particle reveals that the particle consists mainly of \ce{Fe}. Iron is a potential contaminate in polishing grit slurry, but it could also originate in cross-contamination from the polishing of other semiconductors, i.e. \ce{InP} \citep{benson2015as-received}.

%%=====
\subsection{Circular stains}
Four typical stains that can be observed on the substrate surface with \ac{sem} is shown in Fig.~\ref{fig:subB_stains}. One type of stain can be observed in Fig.~\ref{fig:EDX_C-3895-23Ad_m001} and Fig.~\ref{fig:EDX_C-3895-23A_edx1_m005}. One type of stain has a bright background with a darker centre in one part of the stain, as seen in Fig.~\ref{fig:EDX_C-3895-23Ad_m001}--\subref{fig:EDX_C-3895-23A_edx1_m005}. The size of the bright stains varies from \SI{30}{\micro\metre} to \SI{150}{\micro\metre}. The density of this type of stain was estimated from the \ac{sem} grid map to be \SI{2e2}{\centi\metre^{-2}}. These stains could be residue from the evaporation of a droplet on the surface, with the centre consisting of the impurities that were carried by the surface tension of the droplet. %The bright background could be a thin layer of ...

The stain seen in Fig.~\ref{fig:EDX_C-3895-23A_edx1_m003} appears dark in the \ac{sem} image and stains similar to this have sizes from \SI{8}{\micro\metre} to \SI{15}{\micro\metre}. The density of these stains was estimated from the \ac{sem} grid map to be \SI{1e3}{\centi\metre^{-2}}. The last type of stain appears as a dark shadow on the substrate surface when observed in \ac{sem}, see Fig.~\ref{fig:EDX_C-3895-23A_edx1_m016}. The typical size of stains like this is \SI{10}{} to \SI{50}{\micro\metre} and density of these stains was estimated to be \SI{1e4}{\centi\metre^{-2}}. The observed stains have in common that they do not contribute with any additional signal to the \ac{eds} spectrum due to their thin layer on the surface, and it has therefore not been possible to identify what they are. 
%Page 11 in eds report 1-10
\begin{figure}[htbp]
    \centering
    \mySubfigure{0.48\linewidth}{C-3895-23A_edx1_m001.png}[fig:EDX_C-3895-23Ad_m001]
    \mySubfigure{0.48\linewidth}{C-3895-23A_edx1_m005.png}[fig:EDX_C-3895-23A_edx1_m005]
    \par\bigskip
    \mySubfigure{0.48\linewidth}{C-3895-23A_edx1_m003.png}[fig:EDX_C-3895-23A_edx1_m003]
    \mySubfigure{0.48\linewidth}{C-3895-23A_edx1_m016.png}[fig:EDX_C-3895-23A_edx1_m016]
    \caption[\Ac{sem} images of stains on substrate B.]{\Ac{sem} images of \subref{fig:EDX_C-3895-23Ad_m001}--\subref{fig:EDX_C-3895-23A_edx1_m005} bright and \subref{fig:EDX_C-3895-23A_edx1_m003}--\subref{fig:EDX_C-3895-23A_edx1_m016} dark stains on substrate B.}
    \label{fig:subB_stains}
\end{figure}

%%=========================================
%\section{AFM Study of As-Received Substrate B}
\subsection{Surface Scratches and Roughness}
The surface of substrate B has been subjected to a rough polish, and scratches stemming from the polishing can be seen on the surface. The scratches are typically between \SI{10}{\nano\metre} and \SI{100}{\nano\metre} wide, as seen in Fig.~\ref{fig:C-3895-23Ad_m002}, but some large scratches located near the edges are as wide as \SI{1}{\micro\metre}, as seen in Fig.~\ref{fig:C-3895-23A_J08_detail}. The latter are not evenly distributed as the polishing scratches, and they are typical of handling tools, i.e. the teflon tweezers. The surface scratches on substrate B are most likely deeper than those on substrate A since they are visible on the dark field images of substrate B. Substrate B needs a fine polishing before growth to get rid of the scratches.
\begin{figure}[htbp]
    \centering
    \mySubfigure{0.48\linewidth}{C-3895-23Ad_m002.png}[fig:C-3895-23Ad_m002]
    \mySubfigure{0.48\linewidth}{C-3895-23A_J08_detail.png}[fig:C-3895-23A_J08_detail]
    \caption[\Ac{sem} images of scratches on substrate B.]{\Ac{sem} images of \subref{fig:C-3895-23Ad_m002} polishing scratches and \subref{fig:C-3895-23A_J08_detail} deep scratches on substrate B.}
    \label{fig:SEM_C389523_scratches}
\end{figure}

Complementary \ac{afm} images are shown in Fig.~\ref{fig:subBa_afm} where $\SI{5}{\micro\metre}\times\SI{5}{\micro\metre}$ areas were measured at three different locations on the substrate surface: near the centre, near the upper edge, and near the upper left corner. The \ac{rms} roughness of substrate B is \SI{\sim 4}{\nano\metre} at the centre and \SI{\sim 5}{\nano\metre} around the edges and corners, which is a factor \SIrange{12}{16}{} larger than on substrate A. This indicates that the substrate has large scratches and is inferior to substrate A. With too large \ac{rms} roughness, the surface starts looking 3-dimensional instead of 2-dimensional, resulting in poorer film growth (R. Haakenaasen, personal communication, May 29, 2017). The \ac{czt} substrate surfaces are easily damaged by surface scratches caused by mechanical lapping \citep{egan2009scanning}. As can be observed in all the \ac{afm} images, the final polishing step has left scratches on the surface. The largest polishing scratches on substrate B are \SI{0.3}{\micro\metre} wide and \SI{15}{\nano\metre} deep.

Images of , as seen in Fig.~\ref{fig:subAa_afm}. The \ac{rms} roughness of substrate A is \SI{\sim0.3}{\nano\metre} at both the centre and around the edges, while it is slightly higher at the corner with \iac{rms} roughness of \SI{\sim0.4}{\nano\metre}. This indicates the absence of large scratches. The typical surface scratches are between \SIrange{10}{20}{\nano\metre} wide and \SI{1}{\nano\metre} deep. While the largest polishing scratches are \SI{0.2}{\micro\metre} wide and \SI{5}{\nano\metre} deep.

\begin{figure}[htbp]
    \centering
    \begin{subfigure}[b]{0.032\linewidth}
        \label{fig:subBa_afm_scale}\captionsetup{list=no}
        \includegraphics[width=\linewidth]{subBa_afm_scale.png}
    \end{subfigure}
    \hfill
    \mySubfigure{0.3\linewidth}{subBa_afm_centre.png}[fig:subBa_afm_centre]
    \hfill
    \mySubfigure{0.3\linewidth}{subBa_afm_leftedge.png}[fig:subBa_afm_edge]
    \hfill
    \mySubfigure{0.3\linewidth}{subBa_afm_upperleftcorner.png}[fig:subBa_afm_corner]
    \caption[\Ac{afm} of as-received substrate B.]{\Ac{afm} measurements of the as-received substrate B. Images of $\SI{5}{\micro\metre}\times\SI{5}{\micro\metre}$ areas are taken at three different locations on the substrate surface: \subref{fig:subBa_afm_centre} near the centre, \ac{rms} roughness \SI{3,7}{\nano\metre}; \subref{fig:subBa_afm_edge} near the left edge, \ac{rms} roughness \SI{4,8}{\nano\metre}; and \subref{fig:subBa_afm_corner} near the upper left corner, \ac{rms} roughness \SI{4,8}{\nano\metre}. The bright line near the top of the image is due to the tip losing track of the surface.}\label{fig:subBa_afm}
\end{figure} % AFM, substrate B, as-received.

%%=========================================
%\section{Near-IR of As-Received Substrate B}

%%=========================================
\subsection{Impurity Analysis}

\Ac{eds} impurity analysis was performed on the as-received substrate B. Three locations on the surface -- the centre, the edge, and the corner -- were analysed. The results of this analysis can be seen in Table~\ref{tab:subBa_eds_analysis}. The only elements found above the \ac{eds} detection limit, in addition to \ce{Cd}, \ce{Zn}, and \ce{Te}, were \ce{Al}, \ce{Si}, \ce{C}, and \ce{O}. The relative concentrations of \ce{Cd}, \ce{Zn}, and \ce{Te} had an error of less than one percentage point from the expected value of \SI{48}{\atomic\percent} cadmium, \SI{2}{\atomic\percent} zinc and \SI{50}{\atomic\percent} tellurium. The atomic concentration of aluminium and silicon near the centre of the substrate is slightly lower than near the edge and corner\todo{}.

\begin{table}[htbp]
    \centering
    \caption[\Ac{eds} impurity analysis of the as-received substrate B.]{Results of the \ac{eds} impurity analysis at three different locations on the $30\times30$ \SI{}{\milli\metre^2} as-received (111)B \ac{czt} substrate B (atomic concentration \%). The X-ray signal is acquired from $\SI{1270}{\micro\metre}\times\SI{890}{\micro\metre}$ areas near the centre, upper edge, and upper left corner.}\label{tab:subBa_eds_analysis}
    \begin{tabu} to 1.0\textwidth { X[1.85,r] X[1.125,c] X[1.125,c] X[1.125,c] X[1.125,c] X[1.125,c] X[1.125,c] X[1.125,c] }
    \hline
         & \textbf{\ce{Te}} (at.\%) & \textbf{\ce{Cd}} (at.\%) & \textbf{\ce{Zn}} (at.\%) & \textbf{\ce{Al} } (at.\%) & \textbf{\ce{Si}} (at.\%) & \textbf{\ce{C}} (at.\%) & \textbf{\ce{O}} (at.\%) \\ % \textbf{$X$} (\SI{}{\milli\metre}) &  \textbf{$Y$} (\SI{}{\milli\metre})
        \hline
        Near centre & \SI{45.88}{} & \SI{45.35}{} & \SI{2.13}{} & \SI{0.18}{} & \SI{0.47}{} & \SI{4.59}{} & \SI{1.40}{}  \\ %\SI{15.1}{} & \SI{15.1}{}
        Near edge & \SI{45.84}{} & \SI{45.39}{} & \SI{2.28}{} & \SI{0.21}{} & \SI{0.51}{} & \SI{4.59}{} & \SI{1.18}{}   \\ % \SI{15.1}{} & \SI{29.0}{}
        Near corner & \SI{45.86}{} & \SI{45.45}{} & \SI{2.28}{} & \SI{0.36}{} & \SI{0.49}{} & \SI{4.23}{} & \SI{1.33}{}  \\ %\SI{1.0}{}  & \SI{29.0}{}
         \hline
    \end{tabu}
\end{table}
%%========================================
% FTIR transmission spectra.
\subsection{IR Characterisation}

\todo{Beskriv} Fig.~\ref{fig:subB2a_ftir_spectra} \todo{Free carrier absorption.}

\begin{figure}[htbp]
    \centering
    \mySubfigure{0.60175438596\linewidth}{subB2a_121_ftir_spectra.png}[fig:subB2a_ftir_spectra]
    \hfill
    \mySubfigure{0.37824561403\linewidth}{subB2a_121_ftir_transmission_at_k500cm-1.png}[fig:subB2a_ftir_map_500cm-1]
    \caption[\Ac{ftir} measurements of the as-received substrate B2.]{\Ac{ftir} measurements recorded from a $11\times11$ grid on the as-received $\SI{30}{\milli\metre}\times\SI{30}{\milli\metre}$ (111)B-oriented substrate B2: \subref{fig:subB2a_ftir_spectra} Transmission spectra; \subref{fig:subB2a_ftir_map_500cm-1} transmission map at wavenumber $k=\SI{500}{\centi\metre^{-1}}$ showing the transmittance $T$ in percentage of incoming light that is transmitted through at each grid point. The spikes near $k=\SI{4500}{\centi\metre^{-1}}$ and $k=\SI{5000}{\centi\metre^{-1}}$ are an artefact of the \ac{ftir} instrument and should not be considered.}
\end{figure}

The area of lower transmission form a semicircle in the lower part of substrate B2 as seen in Fig.~\ref{fig:subB2a_ftir_map_500cm-1}. Surprisingly, this semicircle is visible as a brighter area in \ac{sem}, see Fig.~\ref{fig:subB2b_sem_low_transmission}. The semicircle starts \SI{4.11}{\milli\metre} from the left edge, goes around up to about \SI{13.11}{\milli\metre} before it goes down, and ends \SI{1.89}{\milli\metre} from the right edge. The major influence on \ac{se} generation is the topography of the surface. Generally, edges and other pointy parts that are facing the detector produce more \acp{se}, and hence, these parts look brighter than the rest of the image \citep{goldstein2012scanning}. This is not the case for the brighter area since \ac{afm} measurements affirm that the surface is as uniform and smooth as the surrounding substrate. Also, the average atomic number influences the contrast, but \ac{eds} spectra confirms that the low-transmission area and the surrounding substrate have the same composition.

It could be different crystal orientation that cause the contrast in the \ac{sem} images in Fig.~\ref{fig:subB2b_sem_low_transmission}. However, since the films which are grown on this kind of low-transmission areas are not affected, it has likely not a different crystal orientation. \Ac{lpe} growth is particularly sensitive for the crystal orientation of the substrate (E. Selvig, personal communication, May 24, 2017).

\citet{sealy2000mechanism} showed that p-type semiconductors in general appear brighter and that n-type in general appear darker than undoped material in \ac{sem}. The best contrast is achieved for low voltage since then the number of electrons that escape the sample have a stronger dependency on the doping in the sample. \todo{fit with the skrå spektra. Free carrier absorption. Er det n eller p? Kan ses fra kontrast iflg. enkelte artikler. Work function?}

\begin{figure}
    \centering
    \mySubfigure{0.49\linewidth}{subB2b_sem_04_m007.png}[fig:subB2b_sem_low_transmission_left]
    \hfill
    \mySubfigure{0.49\linewidth}{subB2b_sem_04_m005.png}[fig:subB2b_sem_low_transmission_right]
    %\includegraphics[width=1.0\linewidth]{subB2b_sem_05_m002.png}\caption{Left edge}
    %\includegraphics[width=1.0\linewidth]{subB2b_sem_05_m005.png}\caption{Upper edge}
    %\includegraphics[width=1.0\linewidth]{subB2b_sem_05_m003.png}\caption{Right edge}
    \caption[\Ac{sem} image of the semicircle with low \ac{ir} transmission on substrate B2.]{\Ac{sem} image of the \subref{fig:subB2b_sem_low_transmission_left} left and \subref{fig:subB2b_sem_low_transmission_right} right edge of the semicircle with low \ac{ir} transmission in the lower part of substrate B2. The semicircle appears brighter than the rest of the substrate surface.}\label{fig:subB2b_sem_low_transmission}
\end{figure}

%%========================================
