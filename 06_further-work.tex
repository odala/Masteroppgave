%%=========================================
\chapter{Recommendations for Further Work}\label{ch:further-work}
% You should give recommendations to possible extensions to your work. The recommendations should be as specific as possible, preferably with an objective and an indication of a possible approach.
%A further line of study based on concepts studied in this thesis would be a more in depth study of the 
%
%\mycomment{Skriv om ulike overflatebehandlinger, mulighet for å bruke nær-IR mikroskopi.}
% The recommendations may be classified as:
% • Short-term
% • Medium-term
% • Long-term
%%=========================================

Further work will first be to repeat the measurements on an additional (111)B-oriented substrate from vendor A and see if the results from this study are reproducible, both in regards to the low density of particles on the as-received surface and the higher density after etch. The source of the grit particles that appeared on the surface after etch should be located by studying the bevelled edges and underneath the substrate. 

Then, different preparation etch procedures, i.e. shorter etch time or spray etching, should be examined. Seeing that \citet{benson2016analysis} reports that their \ce{Br}:methanol-based \ac{mbe} preparation etch does remove residual \ce{SiO2} polishin grit from the substrates, a procedure that replicate their results should be developed.

%Espen forslag:
%Benson reports lower grit densities after etch than in our work. We should try to replicate their results. It is unknown exactly how they rinse the substrate prior to etching.
%
%Some variations that could be tried are the following: 1) Rinsing the substrates in a spinner. Apply warm solvents and/or warm water onto the substrate. The liquid will then rush across the surface and hopefully remove surface particles. If the liquid is coupled to a megaultrasonic generator the cleaning efficiency will increase (Busnaina and Moumen, 2006). This method may be less effective to remove particles on the sides of the substrate which are not parallell to the substrate surface because the liquid will not flow across the sides. 2) Masking the surface and cleaning the edges of the sample with a q-tip dipped in solvent is another possibility.
% Busnaina, A.A. and Moumen, N., PostCMP Cleaning of ThermalOxide Wafers, Fabtech, 2006, 12th ed., pp. 293–297.

%Finne metode for å bli kvitt partikler fra kantene på vendor A substrat alik at de ikke løses opp i etsen og legger seg ned på substratoverflaten. Metode for å etse uten å legge igjen så mye partikler (1) ha en skillevegg i etsebegeret slik at en kan etse i det ene kammeret og så ta opp av væsken på andre siden av en glassvegg. Tendens til at partikler legger seg i overflaten, 2) noen løsemidler er bedre til å ta opp og vekk partikler enn andre.)

%Studere lav transmisjon og om det gir lysere områder i SEM og alltid eller bare der p-doping. Mørkere n-doping?

Correlate the density of donuts with the density of tellurium precipitates near the surface.

%A possible extension of this work is to do a characterisation of the substrate surface after different surface treatments. Substrate A will need a surface preparation including a surface etch, while substrate B will need a surface preparation including both polishing of the surface and an etch. A 3rd substrate (substrate C) from a different vendor than the two in this study (vendor C) will be obtained and compared with substrate A and substrate B.

%Then, \ac{lpe} growth of \ce{HgCdTe} should be done on the different substrates. The \ce{HgCdTe} film surface, crystallinity and defects should be studied. It would be of interest to measure the correlations between preparation method and defects in the grown \ce{HgCdTe} layer as well as the correlation between growth conditions and defects in the grown \ce{HgCdTe} layer.

%It would be useful to take use of other techniques to get a more complete characterisation of the surface. An \ac{afm} could be used to characterise the surface topography and morphological defects in the \ce{CdZnTe} surface with more detail than has been achieved with dark field microscopy and \ac{sem} images so far. \ac{afm} is a high resolution scanning probe microscopy technique that can detect height differences on the order of \SI{1}{\nano\metre}. It might be of interest to do a surface sensitive impurity analysis, i.e. \ac{xps} somewhere else with higher signal intensity, to get a qualitative and quantitative compositional measurement of the impurity concentrations of the outermost layers of the substrates.

% The future work section is a place for you to explain to your readers where you think the results can lead you. What do you think are the next steps to take? What other questions do your results raise? Do you think certain paths seem to be more promising than others?
% The goal should not be to go into a bunch of details, but instead just a sentence or two explaining each idea.

%%=========================================