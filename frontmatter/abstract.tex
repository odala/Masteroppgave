%%=========================================
% Here you give a summary of your work and your results. This is like a management summary and should be written in a clear and easy language, without many difficult terms and without abbreviations. Everything you present here must be treated in more detail in the main report. You should not give any references to the report in the summary - just explain what you have done and what you have found out. Should NOT be more than two pages!
%%=========================================
\cleardoublepage  % to get the abstract to start on an odd page (right-hand page)
\selectlanguage{english}
\begin{abstract}
In this study, three different \ac{czt} substrates were characterised for surface impurity contamination, particles, and defects. The substrates were examined both as-received and after surface pre-growth preparation. As the final step, a layer of \ac{mct} film was grown on each substrate and examined. It is of great interest to obtain a better understanding of how impurities, defects, and particles on the \ac{czt} substrate surface affect the quality of the grown \ac{mct} film.  %%The \ac{mct} film is used to fabricate high performance photovoltaic \ac{ir} detectors. It is important to grow films with high crystallinity, i.e. few grain boundaries and almost no dislocations, and few defects and impurities to ensure that the detector elements that are processed on the film function properly. Hence, i

Two of the substrates were (111)B-oriented and used for \ac{lpe} growth of \ac{mct}: substrate A from vendor A, generally recognised to fabricate the best \ac{czt} substrates, and substrate B from vendor B, which was compared to vendor A. One additional substrate from vendor B, substrate B2, was studied as well. The last substrate was (211)B-oriented and used for \ac{mbe} growth of \ac{mct}, substrate C from vendor A.  The characterisation consisted of a study by optical microscopy, \ac{sem} with \ac{eds}, \ac{afm}, near-\ac{ir} transmission microscopy, and \ac{ftir}. 
%Three $30\times$\SI{30}{\milli\metre^2} (111)B-oriented substrates from two different vendors and one $15\times$\SI{15}{\milli\metre^2} (211)B-oriented substrate 
%The substrates were then subjected to different surface pre-growth preparations. The third $30\times$\SI{30}{\milli\metre^2} (111)B-oriented substrate was a replacement for the second substrate, which broke during this characterisation step. %The state-of-the-art (111)B-oriented substrate was subjected to \ac{lpe} preparation etch, the (111)B-oriented substrates were subjected to \ac{lpe} preparation polish and etch, while the state-of-the-art (211)B-oriented substrate was subjected to \ac{mbe} preparation etch.

The as-received substrate A had the best surface polish and crystal quality and had almost no particles on the surface. Surprisingly, the preparation etch procedure introduced more particles on the surface. The as-received substrate B had scratches, particles, and voids on the surface, and it had a larger surface roughness than the as-received substrate A by a factor of 10. The roughness decreased after the polishing procedure to about the same value as substrate A. The number of particles also decreased, but it was not as low as on substrate A. While the as-received substrate A had a density of polishing grit less than \SI{2e4}{\centi\metre^{-2}}, the as-received substrate C had an average density of \SI{4e7}{\centi\metre^{-2}}. The polishing grit density was reduced by a factor of 10 after etch.%Furthermore, the voids were indicative of poorer crystal quality than substrate A. The results from the polished substrate B was reproduced on the polished substrate B2.

The surface of the \ac{mct} epilayers that were grown by \ac{lpe} on substrate A and substrate B2 showed wavy structures on the surface with doughnut-shaped defects and large circular defects with a density of \SIrange{1e4}{2e4}{\centi\metre^{-2}} and \SI{\sim 2}{\centi\metre^{-2}} respectively. The doughnut-shaped defects seemed to correlate with the polishing grit on substrate B2, but showed no correlation with the polishing grit on substrate A. The \ac{mct} epilayer that was grown by \ac{mbe} on substrate C had a microvoid density of \SI{1e+05}{\centi\metre^{-2}}, which correlated with polishing grit observed before growth.

\end{abstract}
%%=========================================

%%=========================================
\cleardoublepage  % to get the abstract to start on an odd page (right-hand page)
\selectlanguage{norsk}
\begin{abstract}

I dette studiet ble tre ulike \acf{czt} substrater karakterisert for urenheter, partikler og defekter. Substratene ble karakterisert både som motatt fra leverandør og etter forbehandling før groing. Til slutt ble det grodd ett lag med \acf{mct} film på hvert av substratene og filmen ble karakterisert. Det er av interesse å få en bedre forståelse av hvordan urenheter, defekter og partikler på overflaten av substratet påvirker kvaliteten på den grodde \ac{mct} filmen. %\Ac{mct}-filmer blir brukt til å produsere infrarøde (IR) detektorer med høy ytelse.

To av substratene var (111)B-orienterte og brukt til å gro \ac{mct} ved væskefaseepitaksi (LPE): substrat A fra leverandør A, kjent for å produsere substrater av ypperste kvalitet, og substrat B leverandør A, som ble sammenlignet med leverandør A. Ett ekstra substrat fra leverandør B, substrat B2, ble også studert. Det siste substratet var (211)B-orientert og brukt til å gro \ac{mct} ved molekylstråleepitaksi (MBE), substrat C fra leverandør A. Karakteriseringen ble gjennomført ved bruk av optisk mikroskop, sveipeelektronmikroskop (SEM), energidispersiv røntgenspektroskopi (EDS), atomærkraftmikroskopi (AFM), nær-IR transmisjonsmikroskopi og Fouriertransform infrarød spektroskopi (FTIR).

Substrat A hadde den beste overflatepoleringen og krystallkvaliteten som motatt fra leverandør, og hadde omtrent ingen partikler på overflaten. Overraskende nok hadde substrat A mer partikler på overflaten etter etseprosedyren. Substrat B hadde riper, partikler og hulrom på overflaten, og overflateruheten var 10 ganger større enn for substrat A som motatt. Ruheten minket til omtrent det samme som substrat A etter polering og etsing. Antall partikler minket også, men ble ikke like lavt som for substrat A. Substrat C hadde en gjennomsnittlig tetthet at poleringspartikler på \SI{4e7}{\centi\metre^{-2}} som motatt fra leverandør i motsetning til substrat A som hadde en poleringspartikkeltetthet mindre enn \SI{2e4}{\centi\metre^{-2}} som motatt fra leverandør. Tettheten ble redusert med en faktor 10 etter etsing. %Hulrommene i overflaten antyder at krystallkvaliteten var dårligere enn for substrat A. Resulatet for det polerte substrat B ble reprodusert for the polerte substrat B2. 

Overflaten på de LPE-grodde lagene viste en bølgete struktur typisk for LPE med smultringformede defekter og større sirkulære defekter med en tetthet på henholdsvis \SIrange{1e4}{2e4}{\centi\metre^{-2}} og \SI{\sim 2}{\centi\metre^{-2}}. De smultringformede defektene så ut til å korrelere med poleringspartikler på substrat B2, men viste ingen korrelasjon med poleringspartikler på substrat A. \Ac{mct} filmen som ble grodd på substrat C hadde en tetthet av mikrohulrom på \SI{1e+05}{\centi\metre^{-2}}. Mikrohulrommene korrelerte med partikler observert på overflaten av substratet før groing.

\end{abstract}
%%=========================================
\selectlanguage{english}
%%=========================================