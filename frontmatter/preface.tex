%%=========================================
\chapter*{Preface}%\addcontentsline{toc}{chapter}{Preface}
% Here you give a brief introduction to your work. What it is (e.g. a Master's thesis in RAMS at NTNU as a part of the study program XXX and ...) and when it was carried out (e.g. during the autumn of semester of 2021). If it was carried out for a company, you should mention this and also describe the cooperation with the company. You may also describe how the idea to the project was brought up.
%
This master's thesis is submitted as the final work required for the Master's degree (MSc) in Applied Physics and Mathematics at the Norwegian University of Science and Technology (NTNU). The study presented in this thesis was carried out at the Norwegian Defence Research Establishment (FFI) from January to June 2017 under the supervision of Professor Randi Haakenaasen. 

The study was a continuation of the Specialisation Project in Physics (TFY4510, 15 ECTS), as presented in the author's report titled \inquote{Characterisation of As-Received \ce{CdZnTe} Substrates} \citep{lauten2017characterisation}. Some of the results are reiterated here in order to better present the full scope of the study that has been performed during the final year of my MSc studies. %which involved a characterization of two as-received \ce{CdZnTe} substrates from different vendors

% The thesis is to some extent built on previously obtained results, as presented in the author’s project work titled ’Precipitation of Several Coexisting Strengthening Phases in Aluminium Alloys’, written during the autumn of 2015. Some of the contents in this project work are re-given here, in order to better present the full scope of scientific work performed during the final year of my MSc studies in applied physics and mathematics. The thesis serves a dual purpose; studying the multiphase nature of 2xxx series Al alloys, as well as providing an introduction to the SPED technique. It is intended for readers who are interested with one or both of these topics.
%
%\subsection*{Acknowledgments}
%First of all
I would like to thank Professor Randi Haakenaasen for giving me the possibility to carry out the work presented in this thesis. I am grateful for the valuable discussions and guidance during my work. I would also like to thank Chief Scientist Espen Selvig at FFI for his patient advice and follow up in the development of this work; Senior Scientist Kjell Ove Kongshaug and Technician Laila Trosdahl-Iversen at FFI for demonstrating the preparation methods used before epitaxial growth; and Senior Engineer Torgeir Lorentzen at FFI for giving me training at the atomic force microscope and for generally being helpful in practical matters.
%
%Associate Professor Turid Worren Reenaas at NTNU for taking the time to read my master's thesis and provide useful feedback; 
%%=========================================
\vspace{1.0cm}
\begin{flushright}
    Oda Lauten      \\
    Kjeller, Norway \\
    \month~\year    \\
\end{flushright}
%%=========================================