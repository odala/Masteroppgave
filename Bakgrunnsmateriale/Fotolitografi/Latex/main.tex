%########################### MAIN-fil #################################
%% Malen er laget av Oda Lauten og Jan Gulla.

%==============================
\documentclass[5p]{elsarticle}    	                % 5p gir 2 kolonner pr side. 1p gir 1 kolonne pr side.
\usepackage{packages}
\usepackage{commands}
%==============================

%%%%%%%%%%%%%%%%%%%%%%%%%%%%%%%%%%%%%%%%%%%%%%%%%%%%%%%%%%%%%%%%%%%%%%%%%
\begin{document}

%%%%%%% FRONTMATTER %%%%%%%%%%%%%%%%%%%%%%%%%
\begin{frontmatter}
\title{Photolitography}
\author[fysikk]{Oda Lauten}
\address[fysikk]{Department of Physics, Norwegian University of Science and Technology, N-7491 Trondheim, Norway.}

%%=========================================
\begin{abstract}
In this experiment, a photoresist mask was made on a \ce{CdHgTe} thin film using photolitography and then etched.

%This document contains my report of the procedure, results, and analysis of the photolithography and etching process of fabricating a silicon wafer, along with the conclusions I have drawn from this process. Photolithography is a critical step in the fabrication process as the blueprints of most of our circuit components begin to form during this step. The process involves using a chemical known as photoresist which can be selectively exposed to light to create a mask on top of our oxide layer. This mask of photoresist is used to selectively etch away parts of our oxide, leaving behind the desired pattern for the formation of our devices. The goals for this lab are to use lithography to create our photoresist mask and then etch away a specific pattern of the oxide layer, all the while being careful not to over-develop the photoresist or over-etch the oxide, both of which will result in a loss of resolution in our final etched pattern. 

\end{abstract}
%%=========================================
\end{frontmatter}
%%=========================================
\section{Introduction}


%%=========================================
\section{Theory}
 %transfer a pattern to a photoresist by exposure to ultra violet light through an optical mask.

%%=========================================
\section{Apparatur og prosedyre}
Først må waferet renses. Urenheter som støvpartikler kan hindre lyset fra å eksponere fotoresisten eller føre til at fotoresistlaget ikke blir uniformt. CMT-waferet ble renset i metanol. Deretter ble det skylt i acetone og blåst tørt med nitrogengass. Det er viktig at waferet er tørt siden fuktighet minker adhesjonsevnen til fotoresisten.

Det første steget i selve fotolitografiprosessen er å legge et tynt lag fotoresist på overflaten av waferet. Den flytende fotoresisten dryppes på waferet med én sprøyte og deretter spinnes waferet med en jevnt økende hastighet de ti første sekundene og deretter på maks hastighet de neste fem sekundene. Sentrifugalkraften gjør at fotoresistvæsken flyter ut til sidene hvor den bygger seg opp helt til overflatespenningen blir brutt.

Etter spinning blir waferet bakt 45 sekunder ved 35\celsius og deretter 2 minutter ved 90\celsius. Dette er for å fjerne løsemiddel og gi økt feste mellom fotoresisten og overflaten på waferet.

For å lage et mønster i fotoresisten blir ultrafiolett stråling lyst på fotoresisten gjennom en maske. Avhengig av om fotoresisten er positiv eller negativ vil oppløsningsevnen til fotoresisten bli henholdsvis sterkere eller svakere. I dette tilfellet ble det brukt en positiv fotoresist og en firkløver maske slik at området som ble belyst mellom firkløverbladene fikk økt sin oppløsningsevne.

Når den positive fotoresisten fremkalles vil de eksponerte områdene løses opp. Fremkallervæsken brukt i dette oppsettet var \mycomment{fyll inn her}. Waferet ble beveget fram og tilbake i fremkallervæsken i \mycomment{to minutter} deretter ble den beveget i totalt to minutter i rent vann fordelt jevnt på to etapper i to ulike beger med rent vann. Til slutt ble den liggende i en tredje beholder med rent vann.

Den fjernede fotoresisten brukes som en maske for etsing gjennom tynnfilmen ned til det underliggende substratet. Fotoresisten beskytter mot etsevæsken. Etter at mønsteret er etset, kan resisten fjernes. 




%Safety: The chemicals we dealt with in this lab needed to be handled with extreme care. Hydrofluoric acid from the oxide etch is particularly dangerous in that symptoms of exposure may not be immediately apparent. Photoresist and the developer solution are also dangerous chemicals so caution was used in a steps of the lab.

%%=========================================
\section{Results and Discussion}
%%=========================================


%%=========================================
\section{Conclusion}


%%=========================================
\section{\refname}
\bibliographystyle{abbrv}
\bibliography{referanser}
%%=========================================

\end{document}