\clearpage
%%=========================================
%\section[Surface Analysis of Substrate C with Surface Pre-Growth Preparation]{Surface Analysis of Substrate C with Surface Pre-Growth Preparation%
%    \sectionmark{Surface Analysis of Pre-Growth Substrate C}}\sectionmark{Surface Analysis of Pre-Growth Substrate C}\label{sec:subCb}

\section{Surface Analysis of Substrate C after Etching}
As substrate C was already polished by the vendor, the surface pre-growth preparation simply consisted of a \ce{Br}:methanol etch.  The ultimate goal for the \ac{mbe} preparation etch is to provide a clean, smooth, and well-ordered surface to minimise growth defects. As can be seen in Fig.~\ref{fig:subCb_om_df}, there were many features larger than \SI{0.5}{\micro\metre} on the surface. The density of those features was \SI{\sim 8e3}{\centi\metre^{-1}}. 

\begin{figure}[htbp]
    \centering
    \includegraphics[width=0.8\linewidth]{subCb_om_df_n018_5x.jpg}
    \caption[Dark field optical microscopy image of substrate C with surface pre-growth preparation.]{Dark field optical microscopy image of substrate C with surface pre-growth preparation taken near the centre of the substrate.}\label{fig:subCb_om_df}
\end{figure}

\Ac{sem} shows the surface at a higher magnification and it revealed that there were smaller particles distributed over the surface as well. Fig.~\ref{fig:subCb_sem_typical_centre} shows a typical image of an area near the centre of the substrate. Here the particle density is  \SI{1e+07}{\centi\metre^{-2}}. The highest observed density of particles is counted near the upper right corner to be \SI{3e+07}{\centi\metre^{-2}}, see Fig.~\ref{fig:subCb_sem_typical_corner}. 

\begin{figure}[htbp]
    \mySubfigure{0.49\textwidth}{subCb_sem_04a_m008.png}[fig:subCb_sem_typical_centre]
    \hfill
    \mySubfigure{0.49\textwidth}{subCb_sem_04a_m005.png}[fig:subCb_sem_typical_corner]
    \caption[\Ac{sem} images of typical areas on substrate C with surface pre-growth preparation.]{\Ac{sem} images of \subref{fig:subCb_sem_typical_centre} a typical area near the centre and \subref{fig:subCb_sem_typical_corner} an area with a high density of particles near the edge of substrate C after surface pre-growth preparation.}\label{fig:subCb_sem_typical}
\end{figure}

%%=========================================
\subsection{Particles}

Four different types of particles were found and identified on the surface of the \ce{Br}:methanol etched substrate C, as seen in Fig.~\ref{fig:subCb_sem_w_eds}. \todo{}

\begin{figure}
    \centering
    \begin{subfigure}[t]{\textwidth}
        \caption{}\label{fig:subCb_silica}
          \begin{minipage}[t]{0.43\linewidth}
            \centering
            \includegraphics[width=\linewidth]{subCb_sem_07_m002.png}
          \end{minipage}
          \hfill
          \begin{minipage}[t]{0.43\linewidth}
            \centering
            \includegraphics[width=\linewidth]{subCb_sem_07_m002_eds.png}
          \end{minipage}
          \begin{minipage}[t]{0.11\linewidth}
            \centering
            \atomicTable[&][&][&]
          \end{minipage}
    \end{subfigure}
    \par\bigskip
    \begin{subfigure}[t]{\textwidth}
        \caption{}\label{fig:subCb_Br-etch}
          \begin{minipage}[t]{0.43\linewidth}
            \centering
            \includegraphics[width=\linewidth]{subCb_sem_03_m001.png}
          \end{minipage}
          \hfill
          \begin{minipage}[t]{0.43\linewidth}
            \centering
            \includegraphics[width=\linewidth]{subCb_sem_03_m001_eds.png}
          \end{minipage}
          \begin{minipage}[t]{0.11\linewidth}
            \centering
            \atomicTable[&][&][&]
          \end{minipage}
    \end{subfigure}
    \par\bigskip
    \begin{subfigure}[t]{\textwidth}
        \caption{}\label{fig:subCb_Br-etch2}
          \begin{minipage}[t]{0.43\linewidth}
            \centering
            \includegraphics[width=\linewidth]{subCb_sem_09_m004.png}
          \end{minipage}
          \hfill
          \begin{minipage}[t]{0.43\linewidth}
            \centering
            \includegraphics[width=\linewidth]{subCb_sem_09_m004_eds.png}
          \end{minipage}
          \begin{minipage}[t]{0.11\linewidth}
            \centering
            \atomicTable[&][&][&]
          \end{minipage}
    \end{subfigure}
    \caption[\Ac{sem} images, \ac{eds} spectra, and \ac{eds} atomic compositions of four different types of particles found on substrate C after surface pre-growth preparation.]{High resolution \ac{sem} images of four different types of particles found on substrate C after surface pre-growth preparation and the corresponding \ac{eds} spectra and atomic compositions: \subref{fig:subCb_silica} Silica (\ce{SiO2}) polishing grit; \subref{fig:subCb_Br-etch} particle with carbon, bromine, nitrogen, and sulphur; \subref{fig:subCb_Br-etch2} particle with carbon, bromine, and oxygen; and \subref{fig:subCb_silica_large} large silica agglomeration.}\label{fig:subCb_sem_w_eds}
\end{figure}
%
\begin{figure}[htbp]
\ContinuedFloat
    \centering
    \begin{subfigure}[t]{\textwidth}
        \caption{}\label{fig:subCb_silica_large}
          \begin{minipage}[t]{0.43\linewidth}
            \centering
            \includegraphics[width=\linewidth]{subCb_sem_03_m010.png}
          \end{minipage}
          \hfill
          \begin{minipage}[t]{0.43\linewidth}
            \centering
            \includegraphics[width=\linewidth]{subCb_sem_03_m010_eds.png}
          \end{minipage}
          \begin{minipage}[t]{0.11\linewidth}
            \centering
            \atomicTable[&][&][&]
          \end{minipage}
    \end{subfigure}
    \captionsetup{list=no}
    \caption{\emph{(continued)}}
\end{figure}

\todo{The alumina polishing grit observed in \ac{sem} are found all over the surface with a tendency of higher density towards the upper right, lower right, and lower left corners. The particle density was found to be between \SI{5e+06}{\centi\metre^{-2}} and \SI{3e+07}{\centi\metre^{-2}}. The mean particle density was \SI{2e+07}{\centi\metre^{-2}} with a standard deviation of \SI{7e+06}{\centi\metre^{-2}}. In comparison, the average density was \SI{4e+07}{\centi\metre^{-2}} before etch. A graphical representation of the particle density at 25 different locations on substrate C can be seen in Fig.~\ref{fig:subCb_densityData}.} The sample Pearson correlation coefficient show that there is a weak negative correlation of \SI{-0.3}{} between the density of polishing grit on the as-received substrate C and the density of polishing grit after surface pre-growth preparation. This is not enough to say that the density before and after preparation is correlated.

\begin{figure}[htbp]
    \centering
    \includegraphics[width=0.8\linewidth]{subCb_densityData.png}
    \caption[Map of the polishing grit density on substrate C after surface pre-growth preparation.]{A map of the polishing grit density at 25 different locations on the $\SI{15}{\milli\metre}\times\SI{15}{\milli\metre}$ substrate C after surface pre-growth preparation. The density measurements were obtained by counting the number of polishing grits in \ac{sem} images covering $\SI{25.4}{\micro\metre}\times\SI{17.8}{\micro\metre}$ areas. In total, \SI{0.005}{\percent} of the substrate surface was measured. The polishing grit density was observed to vary between \todo{\SI{5e+06}{\centi\metre^{-2}} and \SI{3e+07}{\centi\metre^{-2}}}.}
    \label{fig:subCb_densityData}
\end{figure}


%%=========================================
%\section{AFM Study of Etched Substrate C}
\subsection{Surface Roughness}

The \ce{Br}:methanol etched substrate C was characterised for surface topography by \ac{afm}. Images of $\SI{5}{\micro\metre}\times\SI{5}{\micro\metre}$ areas were taken at three different locations on the substrate surface: near the centre, upper edge, and upper left corner, as seen in Fig.~\ref{fig:subCb_afm}. The \ac{rms} roughness was calculated to be \SI{1.4}{\nano\metre}, \SI{1.4}{\nano\metre}, and \SI{2.8}{\nano\metre}, respectively, using Eq.~\ref{eq:rmsroughness}. This was an increase by \SI{50}{\percent} near the centre, indifferent for the edge, and an increase by a factor 2 near the corner in comparison with the substrate before etch.


%This was an increase in roughness by 2 near the centre, 4 near the edge, and 6 near the corner. The centre had a flat surface with particles with size between \SIrange{10}{50}{\nano\metre}, while the edge and corner had a pebbled surface, which made it difficult to see the presence of particles, as seen in Fig.~\ref{fig:subAb_afm_edge}--\oldsubref{fig:subAb_afm_corner}. The surface scratches from before the etch has disappeared, but the presence of polishing particles have increased the roughness of the surface.

\begin{figure}[htbp]
    \centering
    \begin{subfigure}[c]{0.032\linewidth}
        \label{fig:subCb_afm_scale}\captionsetup{list=no}
        \includegraphics[width=\linewidth]{subCb_afm_scale.png}
    \end{subfigure}
    \hfill
    \mySubfigure{0.3\linewidth}{subCb_afm_centre.png}[fig:subCb_afm_centre]  %\SI{0.85}{\nano\metre}}
    \hfill
    \mySubfigure{0.3\linewidth}{subCb_afm_upperedge.png}[fig:subCb_afm_edge]  %\SI{0.77}{\nano\metre}}
    \hfill
    \mySubfigure{0.3\linewidth}{subCb_afm_upperleftcorner.png}[fig:subCb_afm_corner]  %\SI{1,04}{\nano\metre}}
    \caption[\Ac{afm} of substrate C with surface pre-growth preparation.]{\Ac{afm} measurements of substrate C with surface pre-growth preparation. Images of $\SI{5}{\micro\metre}\times\SI{5}{\micro\metre}$ areas are taken at three different locations on the substrate surface: \subref{fig:subCb_afm_centre} near the centre, \ac{rms} roughness \SI{1.4}{\nano\metre}; \subref{fig:subCb_afm_edge} near the upper edge, \ac{rms} roughness \SI{1.4}{\nano\metre}; and \subref{fig:subCb_afm_corner} near the upper left corner, \ac{rms} roughness \SI{2.8}{\nano\metre}.}
    \label{fig:subCb_afm}
\end{figure} % AFM, substrate C, with surface pre-growth preparation.

%%=========================================
\subsection{Impurity Analysis}

\todo{Sammenlign med før.}

\begin{table}[htbp]
    \centering
    \caption[\Ac{eds} impurity analysis of substrate C with surface pre-growth preparation.]{Results of the \ac{eds} impurity analysis at three different locations on the $\SI{15}{\milli\metre}\times\SI{15}{\milli\metre}$ (211)B \ac{czt} substrate C with surface pre-growth preparation (atomic concentration \%). The X-ray signal is acquired from $\SI{1270}{\micro\metre}\times\SI{890}{\micro\metre}$ areas near the centre, upper edge, and upper left corner.}\label{tab:subCb_eds_analysis}
    \begin{tabu} to 1.0\textwidth { X[1.85,r] X[1.125,c] X[1.125,c] X[1.125,c] X[1.125,c] X[1.125,c] X[1.125,c] X[1.125,c] }
    \hline
         & \textbf{\ce{Te}} (at.\%) & \textbf{\ce{Cd}} (at.\%) & \textbf{\ce{Zn}} (at.\%) & \textbf{\ce{Al} } (at.\%) & \textbf{\ce{Si}} (at.\%) & \textbf{\ce{C}} (at.\%) & \textbf{\ce{O}} (at.\%) \\
        \hline
         Near corner & \SI{45.16}{} & \SI{45.26}{} & \SI{1.40}{} & \SI{1.79}{} & \SI{0.54}{} & \SI{5.85}{} & \SI{0}{} \\ %\SI{1.0}{} & \SI{14.0}{}
         Near edge & \SI{44.91}{} & \SI{44.93}{} & \SI{1.43}{} & \SI{2.33}{} & \SI{0.53}{} & \SI{5.86}{} & \SI{0}{} \\ % \SI{7.5}{} & \SI{14.0}{}
         Near centre & \SI{45.02}{} & \SI{44.88}{} & \SI{1.45}{} & \SI{2.14}{} & \SI{0.52}{} & \SI{6.00}{} & \SI{0}{} \\ % \SI{7.5}{} & \SI{7.5}{} 
         \hline
    \end{tabu}
\end{table}

%%=========================================