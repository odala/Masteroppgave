%%=========================================
\chapter{Conclusions}\label{ch:conclusion}
% In this final chapter you should sum up what you have done and which results you have got. You should also discuss your findings, and give recommendations for further work.
% Here, you present a brief summary of your work and list the main results you have got. You should give comments to each of the objectives in Chapter 1 and state whether or not you have met the objective. If you have not met the objective, you should explain why (e.g., data not available, too difficult). This section is similar to the Summary and Conclusions in the beginning of your report, but more detailed—referring to the the various sections in the report.
% Recap what you did. In about one paragraph recap what your research question was and how you tackled it.
%Highlight the big accomplishments. Spend another paragraph explaining the highlights of your results. These are the main results you want the reader to remember after they put down the paper, so ignore any small details.
%Conclude. Finally, finish off with a sentence or two that wraps up your paper. I find this can often be the hardest part to write. You want the paper to feel finished after they read these. One way to do this, is to try and tie your research to the “real world.” Can you somehow relate how your research is important outside of academia? Or, if your results leave you with a big question, finish with that. Put it out there for the reader to think about to.

In order to fabricate high performance photovoltaic \ac{ir} detectors, it is necessary to start with high quality narrow-gap semiconductor material. \Ac{mct} is the material of choice, as it has high absorption, high electron mobility, small lattice mismatch to \ac{czt} substrates, and it covers both the \ac{mwir} and the \ac{lwir} regions. One of the factors that influences the quality of the grown \ac{mct} material is the \ac{czt} substrate, and both the crystal quality and the surface preparation of the substrate is crucial to growing a good film. Therefore, a study of \ac{czt} substrates for \ac{lpe} and \ac{mbe} growth of \ac{mct} film was performed.

One (111)B substrate from vendor A (substrate A), two (111)B substrates from vendor B (substrate B and substrate B2), and one (211)B substrate from vendor A (substrate C) were investigated using bright and dark field microscopy, \ac{sem}, \ac{eds}, \ac{afm}, near-\ac{ir} transmission microscopy, and \ac{ftir}. The substrates were investigated both as-received and after surface pre-growth preparation, except substrate B2, which was not investigated as-received. As the final step, \iac{mct} film was grown on each substrate, except the shattered substrate B, and the same characterisation as was performed on the substrates, was conducted to correlate the number of defects and the type of defects in the grown \ac{mct} layer with the preparation of the substrate.

The as-received substrate A had the best surface polish and crystal quality, and had almost no particles on the surface. However, \ac{eds} spectra showed trace amounts of \ce{Al} and \ce{Si} on the surface even where no particles were seen in \ac{sem}. Surprisingly, substrate A got more particles on the surface after the etch procedure performed at \ac{ffi}, which was undesirable. The particles might be silica grit left on the edges or underneath the substrate by the vendor's preparation method, which were distributed over the surface during etching. The reason it had not been detected earlier was that \ac{lpe} is somewhat forgiving of impurities on the surface, as there is a period of surface melting before the film starts to grow. Nevertheless, it will most likely influence the film quality.

%The as-received substrate A had \SIrange{1}{5}{\nano\metre} deep surface scratches and a surface roughness of \SI{\sim0.3}{\nano\metre}. Polishing grit and \ac{czt} particles were observed on the surface. The density of features \SI{>0.5}{\micro\metre}, primarily \ac{czt} particles, was \SI{4e2}{\centi\metre^{-2}}. There was too few polishing grit particles on the surface to determine a density, but it is assumed to be less than \SI{2e5}{\centi\metre^{-2}}. After a \ce{Br}:methanol etch, the density of features \SI{>0.5}{\micro\metre}, primarily flakes of \ce{SiO2}, increased to \SI{1e3}{\centi\metre^{-2}}. The average polishing grit density was \SI{7e6}{\centi\metre^{-2}}, which was considerably higher than before etching. The surface roughness increased by a factor 2 near the centre, 4 near the edge, and 6 near the corner. \todo{Correlation between polishing grit and doughnut-shaped defects.}
%

The as-received substrate B had scratches, particles, and voids on the surface, and it had a larger surface roughness than the as-received substrate A by a factor of 10. The roughness decreased after the polishing procedure to about the same value as the etched substrate A. The number of particles also decreased, but it was not as low as on substrate A as-received or etched. Furthermore, the voids were indicative of poorer crystal quality than substrate A. Substrate A may therefore have earned the label of state-of-the-art both because of the surface preparation by the vendor and the crystal quality. The results from the polished substrate B was reproduced on the polished substrate B2. The low-transmission area in substrate B2 was visible in the \ac{sem} images by being brighter than the surrounding substrate. A possible explanation to the contrast in \ac{sem} is that the low-transmission area had a higher free carrier concentration, i.e. p-doped, than the surrounding substrate, which resulted in an increased \ac{sem} signal.

While the as-received (111)B-oriented substrate A from the same vendor had a density of polishing grit less than \SI{2e4}{\centi\metre^{-2}} on the surface, the (211)-oriented substrate C had an average density of \SI{4e7}{\centi\metre^{-2}}. It could be the fact that the (211)-oriented surface was not atomically flat, but consisted of (111)-oriented steps, that caused the high density of polishing grit residue. The polishing grit density was reduced by a factor of 10 after the \ce{Br}:methanol etch. The results from the characterisation of substrate C were similar to the results of \citeauthor{benson2016analysis}, but their etch seemed better than the one performed at \ac{ffi}.

Observation of surface morphology of the \ac{mct} epilayers grown by \ac{lpe} on substrate A and substrate B2 showed wavy structures caused by thickness variations in the film, which are typical of \ac{lpe} growth. In addition, doughnut-shaped defects and large circular defects were observed on the surface with a density of \SIrange{1e4}{2e4}{\centi\metre^{-2}} and \SI{\sim 2}{\centi\metre^{-2}} respectively. The doughnut-shaped defects correlated with the polishing grit on substrate B2, but showed no correlation with the polishing grit on substrate A. The \ac{mct} epilayer grown by \ac{mbe} on substrate C had a microvoid density of \SI{1e+05}{\centi\metre^{-2}}. The microvoids correlated with particles observed on the surface of the substrate before growth.

The findings in this thesis confirm that the surface preparation procedure is important for the growth of high-quality \ac{mct} film. Substrate A had the lowest density of particles and defects on the surface and the highest crystal quality, but the surface preparation needs to be improved. Further understanding of how to perfect the substrate surface could be the key to making even better infrared detector elements.

%\todo{Correlate grit -> microvoids or doughnuts. Te precipitates -> HT voids, round defects.}
% A method was found to remove the ...

%Mye partikler etter behandling.

%Absorpsjon i vendor B substrat.

%Metode for å fjerne domener / fletter fra vendor A-substrat.

%Negative / none / positive correlation between polishing grit particles and doughnuts.

%The findings in this report confirm that the TMDCs are strongly interacting, and further understanding of the interactions and control of these systems could pave the way for spintronics for a new generation of multifunctional electronic devices. % Fin overgang til neste kapittel som er "further work".


% Both the perfection of the substrate surface and that of its crystalline structure are essential for the growth of high-quality material. Thus, CdZnTe surface polishing procedures and growth techniques are crucial issues. (J. Zhao, 2004, Correlation of...)