%%=========================================
\chapter{Conclusions}\label{ch:conclusion}
% In this final chapter you should sum up what you have done and which results you have got. You should also discuss your findings, and give recommendations for further work.
% Here, you present a brief summary of your work and list the main results you have got. You should give comments to each of the objectives in Chapter 1 and state whether or not you have met the objective. If you have not met the objective, you should explain why (e.g., data not available, too difficult). This section is similar to the Summary and Conclusions in the beginning of your report, but more detailed—referring to the the various sections in the report.

One (111)B substrate from vendor A (substrate A), two (111)B substrates from vendor B (substrate B and substrate B2), and one (211)B substrate from vendor A (substrate C) were investigated using bright and dark field microscopy, \ac{sem}, \ac{eds}, \ac{afm}, near-\ac{ir} transmission microscopy, and \ac{ftir}. The substrates were investigated both as-received and after surface pre-growth preparation, except substrate B2, which was not investigated as-received. As the final step, \iac{mct} film was grown on each substrate, except the shattered substrate B, and the same characterisation as for the substrates was conducted to correlate the number of defects and the type of defects in the grown \ac{mct} layer with the preparation of the substrate.

The as-received substrate A had \SIrange{1}{5}{\nano\metre} deep surface scratches and a surface roughness of \SI{\sim0.3}{\nano\metre}. Polishing grit and \ac{czt} particles were observed on the surface. The density of features \SI{>0.5}{\micro\metre}, primarily \ac{czt} particles, was \SI{4e2}{\centi\metre^{-2}}. There was too few polishing grit particles on the surface to determine a density, but it is assumed to be less than \SI{2e5}{\centi\metre^{-2}}. After a \ce{Br}:methanol etch, the density of features \SI{>0.5}{\micro\metre}, primarily flakes of \ce{SiO2}, increased to \SI{1e3}{\centi\metre^{-2}}. The average polishing grit density was \SI{7e6}{\centi\metre^{-2}}, which was considerably higher than before etching. The surface roughness increased by a factor 2 near the centre, 4 near the edge, and 6 near the corner. \todo{Correlation between polishing grit and donut-shaped defects.}

% Observation of surface morphology of \ac{mct} eiplayers grown by \ac{lpe} show the wavy structures typical of \ac{lpe} growth. Other surface features observed are donut-shaped defects and large circular defects.

The as-received substrate B...

The \ac{ftir} low-transmission area in substrate B2 was visible in the \ac{sem} images by being brighter than the surrounding substrate. This would be due to a high free carrier concentration, i.e. p-doped.

The as-received substrate C...

% A method was found to remove the ...

%Two substrates from different manufacturers were characterised for polishing damage, impurities, morphological defects, and residual particles. Substrate A was studied using dark field microscopy. A particle and morphological defect density of features \SI{>0.5}{\micro\metre} of \SI{4e2}{\centi\metre^{-2}} at the centre and \SI{1e3}{\centi\metre^{-2}} at the edges of the substrate was observed. Substrate B was studied using dark field microscopy as well. A particle and morphological defect density of features \SI{>0.5}{\micro\metre} of \SI{1e5}{\centi\metre^{-2}} at both the centre and at the edges of the substrate was observed.

%The substrates were characterised using \ac{sem} and \ac{eds}. Substrate A had polishing scratches that were between 10 and \SI{20}{\nano\metre} wide. Pieces of residual polishing grit was found on the surface. An \ac{eds} spectrum of an agglomeration of multiple particles revealed that it was composed of \ce{Al2O3} and \ce{SiO2} polishing grit. In addition, larger particles with size of \SI{\sim 10}{\micro\metre} on the substrate surface were observed. Substrate B had polishing scratches that were between 10 and \SI{100}{\nano\metre} wide. A large amount of voids with sizes ranging from \SI{5}{} to \SI{100}{\micro\metre} were observed on the surface of substrate B. Most of the particles \SI{>1}{\micro\metre} on the surface of substrate B were \ce{CdZnTe} particles with lengths of between \SI{50}{} and \SI{100}{\micro\metre} that could be debris from the cutting of the substrate, but some carbon based particles was also observed with lengths of \SI{25}{\micro\metre}. Most of the particles \SI{<1}{\micro\metre} on the surface of substrate B were residual \ce{Al2O3} and \ce{SiO2} polishing grit with diameter of between \SI{50}{} and \SI{100}{\nano\metre}.  In addition, iron particles, which could be possible remainders of polishing slurry, and particles containing \ce{Na} and \ce{Cl}, which could be possible remainders of polishing slurry cleaner, were observed on the surface of substrate B.

%By comparing the number of defects in a dark field microscopy image, it was found that substrate A had fewer surface scratches and a lower density of particles and morphological defects on the surface than substrate B. 

%The difference between the two substrates is explained by the fact that substrate A has a more careful and exhaustive preparation than substrate B before delivery. Substrate B is roughly polished and the residual polishing grit and other remaining particles on the surface have not been removed as opposed to on the much smoother surface of substrate A, which has a careful polishing before delivery.


%Mye partikler etter behandling.

%Absorpsjon i vendor B substrat.

%Metode for å fjerne domener / fletter fra vendor A-substrat.

%Negative / none / positive correlation between polishing grit particles and donuts.

% Recap what you did. In about one paragraph recap what your research question was and how you tackled it.
%Highlight the big accomplishments. Spend another paragraph explaining the highlights of your results. These are the main results you want the reader to remember after they put down the paper, so ignore any small details.
%Conclude. Finally, finish off with a sentence or two that wraps up your paper. I find this can often be the hardest part to write. You want the paper to feel finished after they read these. One way to do this, is to try and tie your research to the “real world.” Can you somehow relate how your research is important outside of academia? Or, if your results leave you with a big question, finish with that. Put it out there for the reader to think about to.


%The findings in this report confirm that the TMDCs are strongly interacting, and further understanding of the interactions and control of these systems could pave the way for spintronics for a new generation of multifunctional electronic devices. % Fin overgang til neste kapittel som er "further work".


% Both the perfection of the substrate surface and that of its crystalline structure are essential for the growth of high-quality material. Thus, CdZnTe surface polishing procedures and growth techniques are crucial issues. (J. Zhao, 2004, Correlation of...)