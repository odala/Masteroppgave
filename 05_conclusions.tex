%%=========================================
\chapter{Conclusions}\label{ch:conclusion}
% In this final chapter you should sum up what you have done and which results you have got. You should also discuss your findings, and give recommendations for further work.
% Here, you present a brief summary of your work and list the main results you have got. You should give comments to each of the objectives in Chapter 1 and state whether or not you have met the objective. If you have not met the objective, you should explain why (e.g., data not available, too difficult). This section is similar to the Summary and Conclusions in the beginning of your report, but more detailed—referring to the the various sections in the report.

Mye partikler etter behandling.

Absorpsjon i vendor B substrat.

Metode for å fjerne domener / fletter fra vendor A-substrat.

Negative / none / positive correlation between polishing grit particles and donuts.

% Recap what you did. In about one paragraph recap what your research question was and how you tackled it.
%Highlight the big accomplishments. Spend another paragraph explaining the highlights of your results. These are the main results you want the reader to remember after they put down the paper, so ignore any small details.
%Conclude. Finally, finish off with a sentence or two that wraps up your paper. I find this can often be the hardest part to write. You want the paper to feel finished after they read these. One way to do this, is to try and tie your research to the “real world.” Can you somehow relate how your research is important outside of academia? Or, if your results leave you with a big question, finish with that. Put it out there for the reader to think about to.


%The findings in this report confirm that the TMDCs are strongly interacting, and further understanding of the interactions and control of these systems could pave the way for spintronics for a new generation of multifunctional electronic devices. % Fin overgang til neste kapittel som er "further work".


% Both the perfection of the substrate surface and that of its crystalline structure are essential for the growth of high-quality material. Thus, CdZnTe surface polishing procedures and growth techniques are crucial issues. (J. Zhao, 2004, Correlation of...)