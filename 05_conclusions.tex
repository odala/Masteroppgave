%%=========================================
\chapter{Conclusions}\label{ch:conclusion}
% In this final chapter you should sum up what you have done and which results you have got. You should also discuss your findings, and give recommendations for further work.
% Here, you present a brief summary of your work and list the main results you have got. You should give comments to each of the objectives in Chapter 1 and state whether or not you have met the objective. If you have not met the objective, you should explain why (e.g., data not available, too difficult). This section is similar to the Summary and Conclusions in the beginning of your report, but more detailed—referring to the the various sections in the report.
% Recap what you did. In about one paragraph recap what your research question was and how you tackled it.
%Highlight the big accomplishments. Spend another paragraph explaining the highlights of your results. These are the main results you want the reader to remember after they put down the paper, so ignore any small details.
%Conclude. Finally, finish off with a sentence or two that wraps up your paper. I find this can often be the hardest part to write. You want the paper to feel finished after they read these. One way to do this, is to try and tie your research to the “real world.” Can you somehow relate how your research is important outside of academia? Or, if your results leave you with a big question, finish with that. Put it out there for the reader to think about to.

In order to fabricate high performance photovoltaic \ac{ir} detectors, it is necessary to start with high quality narrow-gap semiconductor material. \Ac{mct} is the material of choice, as it has high absorption, high electron mobility, very small lattice mismatch 
to \ac{czt} substrates, and it covers both the \ac{mwir} and \ac{lwir} regions. One of the factors that influences the quality of the grown \ac{mct} material is the \ac{czt} substrate, and both the crystal quality and the surface preparation of the substrate is crucial to growing a good film. Therefore, a study of \ac{czt} substrates for \ac{lpe} and \ac{mbe} growth of \ac{mct} film have been performed.

One (111)B substrate from vendor A (substrate A), two (111)B substrates from vendor B (substrate B and substrate B2), and one (211)B substrate from vendor A (substrate C) were investigated using bright and dark field microscopy, \ac{sem}, \ac{eds}, \ac{afm}, near-\ac{ir} transmission microscopy, and \ac{ftir}. The substrates were investigated both as-received and after surface pre-growth preparation, except substrate B2, which was not investigated as-received. As the final step, \iac{mct} film was grown on each substrate, except the shattered substrate B, and the same characterisation as for the substrates was conducted to correlate the number of defects and the type of defects in the grown \ac{mct} layer with the preparation of the substrate.

The as-received substrate A had the best surface polish and crystal quality, and had almost no particles on the surface. However, \ac{eds} spectra show trace amounts of \ce{Al} and \ce{Si} on the surface even where no particles were seen in \ac{sem}. Surprisingly, substrate A got more particles on the surface after the etch procedure done at \ac{ffi}. This was undesirable. These may come from the silica grit particles left on the edges or underneath the substrate by the vendor's preparation method. The reason it has not been detected earlier is that \ac{lpe} is somewhat forgiving of impurities on the surface as there is a period of surface melting before the film starts to grow. Nevertheless, it will most likely influence the film quality.

%The as-received substrate A had \SIrange{1}{5}{\nano\metre} deep surface scratches and a surface roughness of \SI{\sim0.3}{\nano\metre}. Polishing grit and \ac{czt} particles were observed on the surface. The density of features \SI{>0.5}{\micro\metre}, primarily \ac{czt} particles, was \SI{4e2}{\centi\metre^{-2}}. There was too few polishing grit particles on the surface to determine a density, but it is assumed to be less than \SI{2e5}{\centi\metre^{-2}}. After a \ce{Br}:methanol etch, the density of features \SI{>0.5}{\micro\metre}, primarily flakes of \ce{SiO2}, increased to \SI{1e3}{\centi\metre^{-2}}. The average polishing grit density was \SI{7e6}{\centi\metre^{-2}}, which was considerably higher than before etching. The surface roughness increased by a factor 2 near the centre, 4 near the edge, and 6 near the corner. \todo{Correlation between polishing grit and donut-shaped defects.}

% Observation of surface morphology of \ac{mct} eiplayers grown by \ac{lpe} show the wavy structures typical of \ac{lpe} growth. Other surface features observed are donut-shaped defects and large circular defects.

The as-received substrate B had lots of scratches, particles, and voids on the surface. The roughness decreased after the polishing procedure to about the same value as substrate A. The number of particles also decreased, but it was not as low as on substrate A as-received or etched. Furthermore, the voids were indicative of poorer crystal quality than substrate A. Substrate A may therefore have earned the label of state-of-the-art both beacuse of the surface preparation by the vendor and the crystal quality. The results from the polished substrate B was reproduced on the polished substrate B2.

The \ac{ftir} low-transmission area in substrate B2 was visible in the \ac{sem} images by being brighter than the surrounding substrate. This would be due to a high free carrier concentration, i.e. p-doped.

%The as-received substrate C was surprisingly full of particles \todo{density instead}. 
While the as-received (111)B-oriented substrate A from the same vendor had no particles on the surface, the (211)-oriented substrate C was full of particles \todo{density instead}. Are all the particles due to the fact that the (211)-oriented surface is not atomically flat, but consists of (111)-oriented steps? \todo{Flere resuktater.} Results on substrate C were the same as Benson, but their etch seems better than ours. We must try their etch procedure and see if that improves the surface.

The findings in this report confirm that the surface preparation procedure is important for the growth of high-quality \ac{mct} film. Substrate A had the lowest density of particles and defects on the surface and the highest crystal quality, but the surface preparation needs to be improved. Further understanding of how to perfect the substrate surface could be the key to making even better infrared detector elements.

\todo{Correlate grit -> microvoids or donuts. Te precipitates -> HT voids, round defects.}
% A method was found to remove the ...

%Mye partikler etter behandling.

%Absorpsjon i vendor B substrat.

%Metode for å fjerne domener / fletter fra vendor A-substrat.

%Negative / none / positive correlation between polishing grit particles and donuts.




%The findings in this report confirm that the TMDCs are strongly interacting, and further understanding of the interactions and control of these systems could pave the way for spintronics for a new generation of multifunctional electronic devices. % Fin overgang til neste kapittel som er "further work".


% Both the perfection of the substrate surface and that of its crystalline structure are essential for the growth of high-quality material. Thus, CdZnTe surface polishing procedures and growth techniques are crucial issues. (J. Zhao, 2004, Correlation of...)